%Imports and customization
\usepackage{tikz}
\usepackage{graphicx}
\usepackage{tikz-feynman}
\usepackage{ulem}
\usepackage{colortbl}
\usepackage{xspace}
\usepackage{caption}
\usepackage{subcaption}
\usepackage{mathrsfs}
\usepackage{multirow}
\graphicspath{ {./figures/} }

\beamertemplatenavigationsymbolsempty
\setbeamertemplate{sidebar right}{}
\setbeamertemplate{footline}{
    \hfill\usebeamertemplate***{navigation symbols}
    \hspace{1cm}\insertframenumber{}/\ref{conclusion}
    %\hspace{1cm}\insertframenumber{}/\inserttotalframenumber
}
\setbeamertemplate{caption}{\raggedright\insertcaption\par}
\setbeamersize{text margin left=4mm,text margin right=4mm} 

\setbeamerfont{itemize/enumerate body}{size=\scriptsize}
\setbeamerfont{itemize/enumerate subbody}{size=\scriptsize}
\setbeamerfont{itemize/enumerate subsubbody}{size=\scriptsize}

%Custom Macros

% WARNING: When using these commands, the image argument must
% NOT have spaces between itself and the braces
\newcommand{\fullscreenimage}[2]{
    \frame{
        \frametitle{#1} 
        \begin{figure}
        \includegraphics[height=0.9\textheight,width=\textwidth,keepaspectratio]{#2}
        \end{figure}
    }
}

\newcommand{\tinyitemize}[1]{
    \begin{itemize}
    {\tiny
        #1
    }
    \end{itemize}
}


\newcommand{\importpdf}[3]{
    \frame{
        \begin{columns}\column{\dimexpr\paperwidth-10pt}
        \begin{figure}
        \includegraphics[page=#2,height=0.8\textheight,width=\textwidth,keepaspectratio]{#1}
        \end{figure}

        {\tiny #3}
        \end{columns}
    }
}


\newcommand{\displayone}[3]{
    \frame{
        \frametitle{#1} 
        \begin{columns}
            \begin{column}{0.5\textwidth}
                #2
            \end{column}
            \begin{column}{0.5\textwidth}
                \begin{figure}
                    \includegraphics[width=\linewidth,height=\textheight,keepaspectratio]{#3}
                \end{figure}
            \end{column}
        \end{columns}
    }
}

\newcommand{\displayonelarge}[3]{
    \frame{
        \frametitle{#1} 
        \begin{columns}
            \begin{column}{0.3\textwidth}
                #2
            \end{column}
            \begin{column}{0.7\textwidth}
                \begin{figure}
                    \includegraphics[width=\linewidth,height=\textheight,keepaspectratio]{#3}
                \end{figure}
            \end{column}
        \end{columns}
    }
}

\newcommand{\displayonecenter}[3]{
    \frame{
        \frametitle{#1} 
        #2
        \begin{figure}
            \includegraphics[height=0.7\textheight,keepaspectratio]{#3}
        \end{figure}
    }
}


\newcommand{\displaytwo}[4]{
    \frame{
        \frametitle{#1} 
        #2
        \begin{columns}
            \begin{column}{0.5\textwidth}
                \begin{figure}
                    \includegraphics[width=\linewidth,height=\textheight,keepaspectratio]{#3}
                \end{figure}
            \end{column}
            \begin{column}{0.5\textwidth}
                \begin{figure}
                    \includegraphics[width=\linewidth,height=\textheight,keepaspectratio]{#4}
                \end{figure}
            \end{column}
        \end{columns}
    }
}

\newcommand{\displaytwoV}[4]{
    \frame{
        \begin{columns}[T]
            \begin{column}{0.5\textwidth}
                \frametitle{#1} 
                #2
            \end{column}
            \begin{column}{0.5\textwidth}
                \begin{figure}
                    \includegraphics[height=0.35\textheight,keepaspectratio]{#3}
                \end{figure}

                \begin{figure}
                    \includegraphics[height=0.35\textheight,keepaspectratio]{#4}
                \end{figure}
            \end{column}
        \end{columns}
    }
}

\newcommand{\displaytwocaption}[6]{
    \frame{
        \frametitle{#1} 
        #2
        \begin{columns}
            \begin{column}{0.5\textwidth}
                \begin{figure}
                    \includegraphics[width=\linewidth,height=\textheight,keepaspectratio]{#3}
                \end{figure}
                \begin{center}
                {\footnotesize #4}
                \end{center}
            \end{column}
            \begin{column}{0.5\textwidth}
                \begin{figure}
                    \includegraphics[width=\linewidth,height=\textheight,keepaspectratio]{#5}
                \end{figure}
                \begin{center}
                {\footnotesize #6}
                \end{center}
            \end{column}
        \end{columns}
    }
}


\newcommand{\displaythree}[5]{
    \frame{
        \begin{columns}[T]
            \begin{column}{0.4\textwidth}
                {\usebeamercolor[fg]{title} \insertframetitle{#1} }\\
                \vspace{5mm}
                #2
            \end{column}
            \begin{column}{0.4\textwidth}
                \begin{figure}
                    \includegraphics[width=\linewidth,height=\textheight,keepaspectratio]{#3}
                \end{figure}
            \end{column}
        \end{columns}
        \begin{columns}[T]
            \begin{column}{0.4\textwidth}
                \begin{figure}
                    \includegraphics[width=\linewidth,height=\textheight,keepaspectratio]{#4}
                \end{figure}
            \end{column}
            \begin{column}{0.4\textwidth}
                \begin{figure}
                    \includegraphics[width=\linewidth,height=\textheight,keepaspectratio]{#5}
                \end{figure}
            \end{column}
        \end{columns}
    }
}

\newcommand{\displaythreecaption}[8]{
    \frame{
        \begin{columns}[T]
            \begin{column}{0.4\textwidth}
                {\usebeamercolor[fg]{title} \insertframetitle{#1} }\\
                \vspace{5mm}
                #2
            \end{column}
            \begin{column}{0.4\textwidth}
                \begin{figure}
                    \includegraphics[width=\linewidth,height=0.35\textheight,keepaspectratio]{#3}
                    \caption{\tiny #4}
                \end{figure}
            \end{column}
        \end{columns}
        \begin{columns}[T]
            \begin{column}{0.4\textwidth}
                \begin{figure}
                    \includegraphics[width=\linewidth,height=0.35\textheight,keepaspectratio]{#5}
                    \caption{\tiny #6}
                \end{figure}
            \end{column}
            \begin{column}{0.4\textwidth}
                \begin{figure}
                    \includegraphics[width=\linewidth,height=0.35\textheight,keepaspectratio]{#7}
                    \caption{\tiny #8}
                \end{figure}
            \end{column}
        \end{columns}
    }
}


\newcommand{\displaythreeseq}[7]{
    \frame{
        \begin{columns}[T]
            \begin{column}{0.4\textwidth}
                {\usebeamercolor[fg]{title} \insertframetitle{#1} }\\
                \vspace{5mm}
                #2
            \end{column}
            \begin{column}{0.4\textwidth}
                \begin{figure}
                    \includegraphics[width=\linewidth,height=\textheight,keepaspectratio]{#3}
                \end{figure}
            \end{column}
        \end{columns}
        \begin{columns}[T]
            \begin{column}{0.4\textwidth}
                \vspace{40mm}
            \end{column}
            \begin{column}{0.4\textwidth} \end{column}
        \end{columns}
    }
    \frame{
        \begin{columns}[T]
            \begin{column}{0.4\textwidth}
                {\usebeamercolor[fg]{title} \insertframetitle{#1} }\\
                \vspace{5mm}
                #4
            \end{column}
            \begin{column}{0.4\textwidth}
                \begin{figure}
                    \includegraphics[width=\linewidth,height=\textheight,keepaspectratio]{#3}
                \end{figure}
            \end{column}
        \end{columns}
        \begin{columns}[T]
            \begin{column}{0.4\textwidth}
                \begin{figure}
                    \includegraphics[width=\linewidth,height=\textheight,keepaspectratio]{#5}
                \end{figure}
            \end{column}
            \begin{column}{0.4\textwidth} \end{column}
        \end{columns}
    }
    \frame{
        \begin{columns}[T]
            \begin{column}{0.4\textwidth}
                {\usebeamercolor[fg]{title} \insertframetitle{#1} }\\
                \vspace{5mm}
                #6
            \end{column}
            \begin{column}{0.4\textwidth}
                \begin{figure}
                    \includegraphics[width=\linewidth,height=\textheight,keepaspectratio]{#3}
                \end{figure}
            \end{column}
        \end{columns}
        \begin{columns}[T]
            \begin{column}{0.4\textwidth}
                \begin{figure}
                    \includegraphics[width=\linewidth,height=\textheight,keepaspectratio]{#5}
                \end{figure}
            \end{column}
            \begin{column}{0.4\textwidth}
                \begin{figure}
                    \includegraphics[width=\linewidth,height=\textheight,keepaspectratio]{#7}
                \end{figure}
            \end{column}
        \end{columns}
    }
}


\newcommand{\displayfour}[5]{
    \frame{
        \frametitle{#1} 
        \begin{columns}[T]
            \begin{column}{0.4\textwidth}
                \begin{figure}
                    \includegraphics[width=\linewidth,height=\textheight,keepaspectratio]{#2}
                \end{figure}
            \end{column}
            \begin{column}{0.4\textwidth}
                \begin{figure}
                    \includegraphics[width=\linewidth,height=\textheight,keepaspectratio]{#3}
                \end{figure}
            \end{column}
        \end{columns}
        \begin{columns}[T]
            \begin{column}{0.4\textwidth}
                \begin{figure}
                    \includegraphics[width=\linewidth,height=\textheight,keepaspectratio]{#4}
                \end{figure}
            \end{column}
            \begin{column}{0.4\textwidth}
                \begin{figure}
                    \includegraphics[width=\linewidth,height=\textheight,keepaspectratio]{#5}
                \end{figure}
            \end{column}
        \end{columns}
    }
}

\newcommand{\displayfourCcaption}[7]{
    \frame{
        \frametitle{\small{#1}} 
        \begin{columns}[T]
            \begin{column}{0.4\textwidth}
                #2\vspace{-2mm}
                \begin{figure}
                    \includegraphics[width=\linewidth,height=\textheight,keepaspectratio]{#4}
                \end{figure}
            \end{column}
            \begin{column}{0.4\textwidth}
                #3\vspace{-2mm}
                \begin{figure}
                    \includegraphics[width=\linewidth,height=\textheight,keepaspectratio]{#5}
                \end{figure}
            \end{column}
        \end{columns}
        \begin{columns}[T]
            \begin{column}{0.4\textwidth}
                \begin{figure}
                    \includegraphics[width=\linewidth,height=\textheight,keepaspectratio]{#6}
                \end{figure}
            \end{column}
            \begin{column}{0.4\textwidth}
                \begin{figure}
                    \includegraphics[width=\linewidth,height=\textheight,keepaspectratio]{#7}
                \end{figure}
            \end{column}
        \end{columns}
    }
}


\newcommand{\pstrike}[2]{
    \only<-\the\numexpr#1-1>{#2}
    \only<#1->{\sout{#2}}
}


\newcommand{\announcesection}[1]{
    \section{#1}
    \frame{
        \begin{center}
            {\huge #1} 
        \end{center}
    }
}

\newcommand{\importpdfwpages}[3]{
    \foreach \pageN in {#2,...,#3}{
        \importpdf{#1}{\pageN}{}
    }
}

% Place your own personal commands here. 
% This is not a part of the base template and so will never be overwritten during updates.
\newcommand{\XSec}{\mathcurse{S}}
\newcommand{\xSec}{{\Large \, \mathcurse{s}}}
\newcommand{\wField}{\mathcal{A}}
\newcommand{\Rparam}{\mathcal{R}}
\newcommand{\matel}{\mathcal{M}}
\newcommand{\amp}{\mathcurse{A}}
\newcommand{\invAmp}{\,\mathcurse{M}\,} % invarient amplitude

\newcommand*{\code}[1]{{\exhyphenpenalty=100000\hyphenpenalty=10000\relax\texttt{#1}}}
\newcommand*{\mhh}{\ensuremath{m_{HH}}\xspace}
\newcommand*{\mjj}{\ensuremath{m_{jj}}\xspace}
\newcommand*{\hhbbbb}{\ensuremath{\textrm{HH} \to 4\textrm{b}}\xspace}
\newcommand*{\hhproc}{\ensuremath{\textrm{VBF} \to \textrm{HH}}\xspace}
\newcommand*{\vbfproc}{\ensuremath{\textrm{VBF} \to 4\textrm{b}}\xspace}
\newcommand*{\vbfhhproc}{\ensuremath{\textrm{VBF} \to \textrm{HH} \to 4\textrm{b}}\xspace}
\newcommand*{\qtil}{\ensuremath{\tilde{q}}\xspace}
\newcommand{\HHVV}{HH\textit{VV}}
\newcommand*{\deta}{\ensuremath{\Delta \eta}\xspace}
\newcommand*{\kvv}{\ensuremath{\kappa_{2V}}\xspace}
\newcommand*{\kl}{\ensuremath{\kappa_{\lambda}}\xspace}
\newcommand*{\kv}{\ensuremath{\kappa_{V}}\xspace}
\newcommand*{\ttbar}{\ensuremath{t \bar{t}}\xspace}
\newcommand*{\xwt}{\ensuremath{X_{W_t}}\xspace}

\newcommand{\fkvv}[1]{\kappa_{2V,#1}}
\newcommand{\fkl} [1]{\kappa_{\lambda,#1}}
\newcommand{\fkv} [1]{\kappa_{V,#1}}

\newcommand{\xsec}{\sigma}
\newcommand{\dXsecM}{\frac{d\sigma(\kv,\kl,\kvv)}{d \mhh}}
\newcommand{\dxsec}[1]{\frac{d \sigma}{d #1}}
\newcommand{\dxsecMN}[1]{\frac{d \sigma_#1}{d \mhh}}
\newcommand{\dXsec}{\dxsec{\Omega}}
\newcommand{\ifb}{fb\textsuperscript{-1} }
\newcommand{\Lag}{\mathscr{L}}
\newcommand{\psibar}{\bar{\psi}}
\newcommand{\emc}{\mathbf{e}}
\newcommand{\bra}[1]{\big< #1 \big|}
\newcommand{\ket}[1]{\big| #1 \big>}
\newcommand{\braket}[1]{\big< #1 \big| #1 \big>}
\newcommand{\Tbraket}[2]{\big< #1 \big| #2 \big>}
\newcommand{\braketA}[2]{\big< #1 \big| #2 \big| #1 \big>}
\newcommand{\TbraketA}[3]{\big< #1 \big| #2 \big| #3 \big>}
\newcommand{\antikt}{anti-k\textsubscript{t}\xspace}
\newcommand{\bbar}{b \bar{b}}

\newcommand{\tinymatrix}[1]{ \big(\begin{smallmatrix} #1 \end{smallmatrix}\big) }
\newcommand{\minimatrix}[1]{{\scalebox{0.75}{\mbox{\arraycolsep=0.3\arraycolsep\ensuremath{\begin{pmatrix}#1\end{pmatrix}}}}}}

