\announcesection{Introduction}

% Theory 1: SM
\displayonelarge{The Standard Model of Particle Physics}{
    {\footnotesize
    Description of all (known) matter and interactions.
    ``Forces'' described as interactions mediated by \textit{gauge boson} particles.
    \vspace{5mm}

    Founded on field theories that assume a paradigm of symmetry,
        which require those gauge bosons to be \textit{massless}.
    \vspace{5mm}

    Confoundingly, the W and Z bosons \textit{do} have mass.
    The cornerstone of the SM is in how it resolves this mass...
    }
}{theory/Standard_Model_of_Elementary_Particles}

% Theory 2: Higgs Field
\displayone{The Higgs Mechanism}{
    { \footnotesize
        The Higgs mechanism posits the existence of a ``Higgs Field''
            which induces mass in any particles it interacts with it.
        \vspace{5mm}


        $U(\phi) = - \mu^2 \phi^2 + \lambda \phi^4$.
        %\vspace{2mm}

        %$\phi \to v+h; v\equiv \frac{\mu}{\sqrt{\lambda}}$.
        \vspace{5mm}

        This field was predicted in 1964 to manifest in the form of a scalar particle,
            the Higgs boson, and was detected on July 4, 2012
            (almost 10 years ago to the day!)
            by the CMS and ATLAS experiments.
        \vspace{5mm}

        Some key properties of the Higgs are still poorly understood,
            but could be measured through dedicated analysis...
    }
}{theory/higgspotential}

% VVh, VVHH, HHH, HHHH (backup)
\frame{
    \frametitle{Measuring Electroweak-Higgs Interactions}
    \begin{figure}
    \centering
    \begin{subfigure}{0.32\textwidth} 
        \resizebox{0.9\textwidth}{!}{
\begin{tikzpicture} \begin{feynman}
    \vertex (kv) {$\kv$};
    \vertex [right=of kv] (h) {$h$};
    \vertex [above left=of kv] (vb1) {$V_1$};
    \vertex [below left=of kv] (vb2) {$V_2$};

    \diagram* {
        (vb1) -- [boson] (kv) -- [boson] (vb2),
        (kv) -- [scalar] (h),
    };
\end{feynman} \end{tikzpicture}
}
 
        \caption*{$VVh$\\ 
            {\tiny ATLAS combined\\$\kv=1.05\pm0.04$}
        }
    \end{subfigure}
    \begin{subfigure}{0.32\textwidth}
        \resizebox{0.8\textwidth}{!}{
\begin{tikzpicture} \begin{feynman}
    \vertex (k2v) {$\kvv$};
    \vertex [above right=of k2v] (h1) {$h_1$};
    \vertex [below right=of k2v] (h2) {$h_2$};
    \vertex [above left=of k2v] (vb1) {$V_1$};
    \vertex [below left=of k2v] (vb2) {$V_2$};

    \diagram* {
        (vb1) -- [boson] (k2v) -- [boson] (vb2),
        (h1) -- [scalar] (k2v) -- [scalar] (h2),
    };
\end{feynman} \end{tikzpicture}
}

        \caption*{$VVhh$\\
            {\tiny ATLAS Combined\\$-0.76 < \kvv < 2.90$\\$(-0.91 < \kvv < 3.11)$}
        }
    \end{subfigure}
    \begin{subfigure}{0.32\textwidth}
        \resizebox{0.9\textwidth}{!}{
\begin{tikzpicture} \begin{feynman}
    \vertex (h0) {$h_0$};
    \vertex [right=of h0] (kl) {$\kl$};
    \vertex [above right=of kl] (h1) {$h_1$};
    \vertex [below right=of kl] (h2) {$h_2$};

    \diagram* {
        (h0) -- [scalar] (kl),
        (h1) -- [scalar] (kl) -- [scalar] (h2),
    };
\end{feynman} \end{tikzpicture}
}
 
        \caption*{$hhh$\\
            {\tiny Prior VBF\to4b\\$-5.0 < \kl < 12.0$\\$(-5.8 < \kl < 12.0)$}
        }
    \end{subfigure}
    \end{figure}

    { \footnotesize
        SM predicts the ``strength'' particle interactions should have,\\
            which is related to how often they occur.
        \vspace{3mm}

        ``$\kappa$-coefficients'' are a method of measuring deviation from the SM,\\
            where e.g. $\kvv \equiv g_{HHVV}^{\textrm{true}} / g_{HHVV}^{\textrm{theory}} \quad;\quad
                g_{HHVV}^{\textrm{theory}}\equiv 4\lambda\frac{m_V^2}{m_h^2}$
        \vspace{3mm}

        By providing the conditions necessary for these interactions,
            one can measure how often they actually occur vs their theoretically predicted rate.
    }
}


% Theory 3: di-Higgs and VBF->HH->bbbb
\frame{
    \frametitle{Conditions for Electroweak-Higgs Interaction}
    \framesubtitle{Vector Boson Fusion, di-Higgs Production, and the All-Hadronic Final State}
    \begin{columns}
        \begin{column}{0.4\textwidth}
            {\footnotesize
                $V$ and $h$ are both unstable,
                    need to choose production (VBF) and decay (4b)
            }

            \resizebox{.8\textwidth}{!}{
\begin{tikzpicture} \begin{feynman}
    \vertex (c2v) {???};
    \vertex [above right=of c2v] (h1) {$h_1$};
    \vertex [below right=of c2v] (h2) {$h_2$};
    \vertex [above left=of c2v] (vb1);
    \vertex [below left=of c2v] (vb2);
    \vertex [left=of vb1] (q1) {$q_1$};
    \vertex [left=of vb2] (q2) {$q_2$};
    \vertex [above right=of h1] (b1) {$b$};
    \vertex [below right=of h2] (bbar2) {$\bar b$};
    \vertex [below=of b1] (bbar1) {$\bar b$};
    \vertex [above=of bbar2] (b2) {$b$};

    \vertex [above=of b1] (q3) {$q_3$};
    \vertex [below=of bbar2] (q4) {$q_4$};

    \diagram* {
        (q1) -- (vb1) -- (q3),
        (q2) -- (vb2) -- (q4), 
        (vb1) -- [boson] (c2v) -- [boson] (vb2),
        (h1) -- [scalar] (c2v) -- [scalar] (h2),
        (b1) -- (h1) -- (bbar1),
        (b2) -- (h2) -- (bbar2),
    };
\end{feynman} \end{tikzpicture}
}
 
        \end{column}
        \begin{column}{0.6\textwidth}
            \begin{table}
\centering
\scriptsize
\begin{tabular}{|c|l|l|}
    \hline
    Higgs Production &          &                       \\
    Process          &   Value  &  Total                \\
    $(|y_H| < 2.5)$  &    [pb]  &  Uncertainty          \\
    \hline
    ggF              &    46.5  &  $\pm4.0$             \\
    VBF              &    4.25  &  $+0.84 \atop -0.77$  \\
    WH               &    1.57  &  $+0.48 \atop -0.46$  \\
    ZH               &    0.84  &  $+0.25 \atop -0.23$  \\
    $t\bar{t}H+tH$   &    0.71  &  $+0.15 \atop -0.14$  \\
    \hline
\end{tabular}
\end{table}

            \begin{table}
\centering
\scriptsize
\begin{tabular}{|l|l|l|}
    \hline
    Decay channel & Branching ratio & Rel. uncertainty  \\
    \hline
    $ H \to b \bar{b}        $    & $5.82 \times 10^{-1} $    & $ +1.2\% \atop -1.3\% $ \\
    $ H \to W^+ W^-          $    & $2.14 \times 10^{-1} $    & $\pm 1.5\%        $   \\
    $ H \to \tau^+ \tau^-    $    & $6.27 \times 10^{-2} $    & $\pm 1.6\%        $   \\
    $ H \to c \bar{c}        $    & $2.89 \times 10^{-2} $    & $ +5.5\% \atop -2.0\% $ \\
    $ H \to ZZ               $    & $2.62 \times 10^{-2} $    & $\pm 1.5\%        $   \\
    $ H \to \gamma \gamma    $    & $2.27 \times 10^{-3} $    & $    2.1\%        $   \\
    $ H \to Z \gamma         $    & $1.53 \times 10^{-3} $    & $\pm 5.8\%        $  \\
    $ H \to \mu^+ \mu^-      $    & $2.18 \times 10^{-4} $    & $\pm 1.7\%        $  \\
    \hline
\end{tabular}
\end{table}

        \end{column}
    \end{columns}
}


\frame{
    \frametitle{\hhproc Production, Leading Order in SM}

    \begin{figure}
    \centering
    \begin{subfigure}{0.32\textwidth} 
        \resizebox{0.9\textwidth}{!}{
\begin{tikzpicture} \begin{feynman}
    \vertex (kv1) {$\kv$};
    \vertex [below=of kv1] (kv2) {$\kv$};
    \vertex [right=of kv1] (h1) {$h_1$};
    \vertex [right=of kv2] (h2) {$h_2$};
    \vertex [above left=of kv1] (vb1);
    \vertex [below left=of kv2] (vb2);
    \vertex [left=of vb1] (q1) {$q_{1}$};
    \vertex [left=of vb2] (q2) {$q_{2}$};

    \vertex [above=of h1] (q3) {$q_{3}$};
    \vertex [below=of h2] (q4) {$q_{4}$};

    \diagram* {
        (q1) -- (vb1) -- (q3),
        (q2) -- (vb2) -- (q4), 
        (vb1) -- [boson] (kv1) -- [boson] (kv2)-- [boson] (vb2),
        (h1) -- [scalar] (kv1),
        (h2) -- [scalar] (kv2),
    };
\end{feynman} \end{tikzpicture}
}
 
        \caption{$M_t$}
        \label{fig:tree_level_vbfhh:kv}
    \end{subfigure}
    \begin{subfigure}{0.32\textwidth}
        \resizebox{0.9\textwidth}{!}{
\begin{tikzpicture} \begin{feynman}
    \vertex (kv) {$\kv$};
    \vertex [right=of kv] (kl) {$\kl$};
    \vertex [above right=of kl] (h1) {$h_1$};
    \vertex [below right=of kl] (h2) {$h_2$};
    \vertex [above left=of kv] (vb1);
    \vertex [below left=of kv] (vb2);
    \vertex [left=of vb1] (q1) {$q_{1}$};
    \vertex [left=of vb2] (q2) {$q_{2}$};

    \vertex [above=of h1] (q3) {$q_{3}$};
    \vertex [below=of h2] (q4) {$q_{4}$};

    \diagram* {
        (q1) -- (vb1) -- (q3),
        (q2) -- (vb2) -- (q4), 
        (vb1) -- [boson] (kv) -- [boson] (vb2),
        (kv) -- [scalar] (kl),
        (h1) -- [scalar] (kl) -- [scalar] (h2),
    };
\end{feynman} \end{tikzpicture}
}
 
        \caption{$M_s$}
        \label{fig:tree_level_vbfhh:kl}
    \end{subfigure}
    \begin{subfigure}{0.32\textwidth}
        \resizebox{0.8\textwidth}{!}{
\begin{tikzpicture} \begin{feynman}
    \vertex (k2v) {$\kvv$};
    \vertex [above right=of k2v] (h1) {$h_1$};
    \vertex [below right=of k2v] (h2) {$h_2$};
    \vertex [above left=of k2v] (vb1);
    \vertex [below left=of k2v] (vb2);
    \vertex [left=of vb1] (q1) {$q_{1}$};
    \vertex [left=of vb2] (q2) {$q_{2}$};

    \vertex [above=of h1] (q3) {$q_{3}$};
    \vertex [below=of h2] (q4) {$q_{4}$};

    \diagram* {
        (q1) -- (vb1) -- (q3),
        (q2) -- (vb2) -- (q4), 
        (vb1) -- [boson] (k2v) -- [boson] (vb2),
        (h1) -- [scalar] (k2v) -- [scalar] (h2),
    };
\end{feynman} \end{tikzpicture}
}
 
        \caption{$M_x$}
        \label{fig:tree_level_vbfhh:k2v}
    \end{subfigure}
    \end{figure}

    \begin{equation} \begin{split}
        \sigma &\propto |  \kv^2 M_t + \kv \kl M_s + \kvv M_x |^2 \\
        \sigma &\propto \kv^2 \kl^2 a_1 + \kv^4 a_2 + \kvv^2 a_3 + \kv^3 \kl a_4 + \kv \kl \kvv a_5 + \kv^2 \kvv a_6
    \end{split} \end{equation}
}

% LHC 1: xsec/lumi
\frame{
    \frametitle{Cross-section and Integrated Luminosity}

    {
    \begin{equation} \begin{split}
        \textrm{Expected Events} &= (\textrm{Probability}) \times (\textrm{Number of repeated trials}) \\
        &\downarrow \\
        \textrm{Expected Events} &= \sigma \times L_{\textrm{integrated}}
        \nonumber
    \end{split} \end{equation}
    }
    \vspace{5mm}

    \begin{center}
    \begin{tabular}{ rll }
    $\sigma$ & $\rightarrow$ & Units of cross-sectional area (femto-barns, $fb = 10^{-43} m^2$);\\
        &&Purely a function of theory\\
    \\
    $L_{\textrm{integrated}}$ & $\rightarrow$ & Units of inverse area ($\ifb$);\\
        && Purely a function of experimental setup 
    \end{tabular}
    \end{center}


}

%\displayone{Cross-section and Luminosity}{
%    {\scriptsize
%    \begin{equation} \begin{split}
%        %\frac{\mathcal{P}_{\textrm{hit}}}{\textrm{throw}} &= \frac{\textrm{Area}_{\textrm{object}}}{\textrm{Area}_{\textrm{total}}} \\
%        %\textrm{expected hits} &= \frac{\mathcal{P}_{\textrm{hit}}}{{\textrm{throw}}} \times N_{\textrm{throws}}
%        %N_{\textrm{throws}} &= \frac{N_{\textrm{throws}}}{\Delta t} \times \Delta t \\
%        %\textrm{expected hits} &= \frac{\mathcal{P}_{\textrm{hit}}}{{\textrm{throw}}} \times N_{\textrm{throws}}
%        \frac{\mathcal{P}_{\textrm{hit}}}{\textrm{throw}} &= \frac{\xsec}{\textrm{Area}_{\textrm{total}}} \\
%        \frac{\textrm{expected hits}}{\Delta t} &= \frac{\mathcal{P}_{\textrm{hit}}}{{\textrm{throw}}} \times \frac{\textrm{throws}}{\textrm{\Delta t}} \\
%        \textrm{expected hits} &= \frac{\textrm{expected hits}}{\Delta t} \times {\Delta t}
%        \nonumber
%    \end{split} \end{equation}
%    \vspace{3mm}
%
%    \begin{equation} \begin{split}
%        \dXsec &\propto | M | ^2 \\
%        \xsec &= \int \dXsec d\Omega \\
%        L_{\textrm{instantaneous}} &= \frac{1}{\textrm{Area}_{\textrm{total}}} \times \frac{N_{\textrm{throws}}}{\Delta t} \\
%        L_{\textrm{integrated}} &= \int L_{\textrm{instantaneous}} dt \\
%        \textrm{expected events} &= \xsec \times L_{\textrm{integrated}}
%        \nonumber
%    \end{split} \end{equation}
%    }
%}{introduction/xsec_ball}

%\frame{
%    \frametitle{Cross-section and Luminosity}
%    \begin{columns}
%        \begin{column}{0.4\textwidth}
%        \end{column}
%        \begin{column}{0.6\textwidth}
%
%        \end{column}
%    \end{columns}
%}

% LHC 1: overview
\displaytwoV{The Large Hadron Collider}{
    \vspace{10mm}
    \begin{itemize}
        \item Constructed 1995-2007
        \begin{itemize}
            \item On border of Switzerland and France
            \item 50-175 meters underground
            \item 26.7 km radius 
        \end{itemize}
        \item Run 2: 2015-2018
        \begin{itemize}
        \item 13 TeV center-of-mass collision energy
        %\item 40 million bunch-crossings per second
        %\item 60 interactions per bunch-crossing
            \item 2.4 billion interactions per second
            \item $L_{\textrm{inst}} = 2\times10^{-5} \textrm{fb}^{-1}\textrm{s}^{-1}$
            \item 139 \ifb of data delivered to P1
            \item 126 \ifb used for this analysis
        \end{itemize}
        \item VBF$\to$H$\to b\bar{b}$ cross-section = 2.5 pb
        \begin{itemize}
            \item 1 VBF$\to$H$\to b\bar{b}$ every 20 seconds %1 higgs at all per second
            \item \vbfhhproc \sigma = 1 fb
            \item 1 \vbfhhproc every 14 \textit{hours}
        \end{itemize}
    \end{itemize}
}{introduction/lhc}{introduction/lhc_pipe}


% ATLAS 1: overview
\displayonelarge{The ATLAS Experiment}{
    {\small
    Massive particle detector array,
        forming a high hermetic, (nearly) $4\pi$ 3D ``camera'' around the particle interaction region
    \vspace{8mm}

    Different subsytems have different roles:
    \begin{itemize}
        \item Tracker = momentum of charged particles
        \item Calorimeter = energy
        \item Muon System = muon tracking
    \end{itemize}
    }
}{atlas/atlas_xsec}
%TODO: muon system is used for bjet E-correction. Look it up
