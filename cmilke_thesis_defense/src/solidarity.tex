%\announcesection{Solidarity}
\displayonelarge{Which Points to Test?}{
    There are a vast number of points to potentially simulate, and I'm unsure which points should be preferred.
    \vspace{5mm}

    Need a way to narrow down the parameter space.
}{solidarity/negative_weights_toprank0}

\frame{
    \frametitle{Closer Look at How Combination is Done}
    {\footnotesize
        The long polynomial functions are just coefficients $c_i(\kvv,\kl,\kv)$ to the x-secs $|A_i|^2 = \sigma_i$
        The linearly combined signal distribution $\tilde{\sigma}(\kvv,\kl,\kv)$ can be viewed more simply as:

    }
    \vspace{5mm}

    $ \tilde{\sigma}(\kvv,\kl,\kv) = $
    \vspace{3mm}
    \begin{columns}
        \begin{column}{0.55\textwidth}
            \resizebox{0.8\textwidth}{!}{ \begin{minipage}{1.0\textwidth}
            {\tiny $
    \left(2 \kappa_{2V}^{2} - \frac{124 \kappa_{2V} \kappa_{V}^{2}}{9} + \frac{61 \kappa_{2V} \kappa_{V} \kappa_{\lambda}}{9} + \frac{106 \kappa_{V}^{4}}{9} - \frac{17 \kappa_{V}^{3} \kappa_{\lambda}}{3} - \frac{\kappa_{V}^{2} \kappa_{\lambda}^{2}}{9}\right) \left|{A{\left(1,1,1 \right)}}\right|^{2} +
$

$
    \left(2 \kappa_{2V}^{2} - 8 \kappa_{2V} \kappa_{V}^{2} + 3 \kappa_{2V} \kappa_{V} \kappa_{\lambda} + 6 \kappa_{V}^{4} - 3 \kappa_{V}^{3} \kappa_{\lambda}\right) \left|{A{\left(2,1,1 \right)}}\right|^{2} +
$

$
    \left(- 4 \kappa_{2V}^{2} + 20 \kappa_{2V} \kappa_{V}^{2} - 8 \kappa_{2V} \kappa_{V} \kappa_{\lambda} - 16 \kappa_{V}^{4} + 8 \kappa_{V}^{3} \kappa_{\lambda}\right) \left|{A{\left(1.5,1,1 \right)}}\right|^{2} +
$

$
    \left(16 \kappa_{2V} \kappa_{V}^{2} - 16 \kappa_{2V} \kappa_{V} \kappa_{\lambda} - 16 \kappa_{V}^{4} + 16 \kappa_{V}^{3} \kappa_{\lambda}\right) \left|{A{\left(0,1,0.5 \right)}}\right|^{2} +
$

$
    \left(\frac{4 \kappa_{2V} \kappa_{V}^{2}}{5} - \frac{4 \kappa_{2V} \kappa_{V} \kappa_{\lambda}}{5} + \frac{\kappa_{V}^{4}}{5} - \frac{3 \kappa_{V}^{3} \kappa_{\lambda}}{10} + \frac{\kappa_{V}^{2} \kappa_{\lambda}^{2}}{10}\right) \left|{A{\left(1,0,1 \right)}}\right|^{2} +
$

$
    \left(- \frac{\kappa_{2V} \kappa_{V}^{2}}{45} + \frac{\kappa_{2V} \kappa_{V} \kappa_{\lambda}}{45} + \frac{\kappa_{V}^{4}}{45} - \frac{\kappa_{V}^{3} \kappa_{\lambda}}{30} + \frac{\kappa_{V}^{2} \kappa_{\lambda}^{2}}{90}\right) \left|{A{\left(1,10,1 \right)}}\right|^{2}
$
}
            \end{minipage}}
        \end{column}
        \begin{column}{0.05\textwidth}
            \rightarrow
        \end{column}
        \begin{column}{0.4\textwidth}
            { \small
                $c_1(\kvv,\kl,\kv) \times \sigma(1,1,1   ) +$\\
                $c_2(\kvv,\kl,\kv) \times \sigma(2,1,1   ) +$\\
                $c_3(\kvv,\kl,\kv) \times \sigma(1.5,1,1 ) +$\\
                $c_4(\kvv,\kl,\kv) \times \sigma(0,1,0.5 ) +$\\
                $c_5(\kvv,\kl,\kv) \times \sigma(1,0,1   ) +$\\
                $c_6(\kvv,\kl,\kv) \times \sigma(1,10,1  )  $\\
            \par }
        \end{column}
    \end{columns}
}


\frame{
    \frametitle{Closer Look at How Combination is Done}

    $ \tilde{\sigma}(\kvv,\kl,\kv) = $

    { \small
        \hspace{20pt} $c_1(\kvv,\kl,\kv) \times \sigma_1 +$\\
        \hspace{20pt} $c_2(\kvv,\kl,\kv) \times \sigma_2 +$\\
        \hspace{20pt} $c_3(\kvv,\kl,\kv) \times \sigma_3 +$\\
        \hspace{20pt} $c_4(\kvv,\kl,\kv) \times \sigma_4 +$\\
        \hspace{20pt} $c_5(\kvv,\kl,\kv) \times \sigma_5 +$\\
        \hspace{20pt} $c_6(\kvv,\kl,\kv) \times \sigma_5  $\\
    \par }
    \vspace{5mm}

    { \footnotesize
        Neither the coefficients nor the xsecs can be looked at in isolation; they must be viewed \textit{together} as a combined product.
        \vspace{3mm}

        If the \textit{magnitude} of some $c_i\sigma_i$ products are disproportionately large for some $\kappa$ value,
        then the smaller $c_i\sigma_i$ (and their associated sample) barely contribute to the combined signal.
        \vspace{3mm}

        The combination depends on \textit{all} six samples.
        All six samples should be contributing at all points (no slackers!).
    \par }
}


\displaytwo{The Metric of Solidarity}{
    Define measure of ``closeness" of $c_i\sigma_i$ terms:
    \vspace{3mm}

    \textit{Solidarity}, $S \equiv \frac{ \sum\limits_{i=1}^6 c_i\sigma_i }{ \textrm{Stdev}(|c_i\sigma_i|) } $
    \vspace{3mm}

     {\tiny
        i.e. take the standard deviation of the \textit{absolute values} of the $c_i\sigma_i$ products,

        normalize this by their sum (the modeled x-sec at that point),

        and then take the \textit{reciprocal} of this normalized standard deviation ($S$ increases as standard deviation gets smaller).
        \par
    }
}{solidarity/negative_weights_toprank0}{solidarity/solidarity_main}

\displayonelarge{Correlation Between Negative Weights and Solidarity}{
    Take the surface integral of the Solidarity map for every basis to produce a ``solidarity integral", and compare to Nweight Integral.
    \vspace{5mm}

    Higher Solidarity values (derived from pure theory) show remarkable correlation to post-selection performance.
}{solidarity/Nweight_integral_VS_theory_solidarity_integral}

\displaytwocaption{Old Samples VS New}{
    Performance of current samples could have been predicted based on solidarity alone.

    (Max of 300 VS the original 400)
}
{solidarity/old_Nweight_integral_VS_theory_solidarity_integral}{Old Samples}
{solidarity/Nweight_integral_VS_theory_solidarity_integral}{New Samples}
