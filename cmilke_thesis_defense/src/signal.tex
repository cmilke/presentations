\announcesection{Signal Modelling}
\frame{
    \frametitle{\hhproc Production, LO in SM}

    \begin{figure}
    \centering
    \begin{subfigure}{0.32\textwidth} 
        \resizebox{0.9\textwidth}{!}{
\begin{tikzpicture} \begin{feynman}
    \vertex (kv1) {$\kv$};
    \vertex [below=of kv1] (kv2) {$\kv$};
    \vertex [right=of kv1] (h1) {$h_1$};
    \vertex [right=of kv2] (h2) {$h_2$};
    \vertex [above left=of kv1] (vb1);
    \vertex [below left=of kv2] (vb2);
    \vertex [left=of vb1] (q1) {$q_{1}$};
    \vertex [left=of vb2] (q2) {$q_{2}$};

    \vertex [above=of h1] (q3) {$q_{3}$};
    \vertex [below=of h2] (q4) {$q_{4}$};

    \diagram* {
        (q1) -- (vb1) -- (q3),
        (q2) -- (vb2) -- (q4), 
        (vb1) -- [boson] (kv1) -- [boson] (kv2)-- [boson] (vb2),
        (h1) -- [scalar] (kv1),
        (h2) -- [scalar] (kv2),
    };
\end{feynman} \end{tikzpicture}
}
 
        \caption{$M_t$}
        \label{fig:tree_level_vbfhh:kv}
    \end{subfigure}
    \begin{subfigure}{0.32\textwidth}
        \resizebox{0.9\textwidth}{!}{
\begin{tikzpicture} \begin{feynman}
    \vertex (kv) {$\kv$};
    \vertex [right=of kv] (kl) {$\kl$};
    \vertex [above right=of kl] (h1) {$h_1$};
    \vertex [below right=of kl] (h2) {$h_2$};
    \vertex [above left=of kv] (vb1);
    \vertex [below left=of kv] (vb2);
    \vertex [left=of vb1] (q1) {$q_{1}$};
    \vertex [left=of vb2] (q2) {$q_{2}$};

    \vertex [above=of h1] (q3) {$q_{3}$};
    \vertex [below=of h2] (q4) {$q_{4}$};

    \diagram* {
        (q1) -- (vb1) -- (q3),
        (q2) -- (vb2) -- (q4), 
        (vb1) -- [boson] (kv) -- [boson] (vb2),
        (kv) -- [scalar] (kl),
        (h1) -- [scalar] (kl) -- [scalar] (h2),
    };
\end{feynman} \end{tikzpicture}
}
 
        \caption{$M_s$}
        \label{fig:tree_level_vbfhh:kl}
    \end{subfigure}
    \begin{subfigure}{0.32\textwidth}
        \resizebox{0.8\textwidth}{!}{
\begin{tikzpicture} \begin{feynman}
    \vertex (k2v) {$\kvv$};
    \vertex [above right=of k2v] (h1) {$h_1$};
    \vertex [below right=of k2v] (h2) {$h_2$};
    \vertex [above left=of k2v] (vb1);
    \vertex [below left=of k2v] (vb2);
    \vertex [left=of vb1] (q1) {$q_{1}$};
    \vertex [left=of vb2] (q2) {$q_{2}$};

    \vertex [above=of h1] (q3) {$q_{3}$};
    \vertex [below=of h2] (q4) {$q_{4}$};

    \diagram* {
        (q1) -- (vb1) -- (q3),
        (q2) -- (vb2) -- (q4), 
        (vb1) -- [boson] (k2v) -- [boson] (vb2),
        (h1) -- [scalar] (k2v) -- [scalar] (h2),
    };
\end{feynman} \end{tikzpicture}
}
 
        \caption{$M_x$}
        \label{fig:tree_level_vbfhh:k2v}
    \end{subfigure}
    \end{figure}

    \begin{equation} \begin{split}
        \sigma &\propto |  \kv^2 M_t + \kv \kl M_s + \kvv M_x |^2 \\
        \sigma &\propto \kv^2 \kl^2 a_1 + \kv^4 a_2 + \kvv^2 a_3 + \kv^3 \kl a_4 + \kv \kl \kvv a_5 + \kv^2 \kvv a_6
    \end{split} \end{equation}
}

\displaytwo{Signal MC}{
    Different values for the $\kappa$'s (\kvv, \kl, \kv) leads to very different kinematic distributions and event yields.
    \vspace{5mm}

    We need large coverage of the 3-coupling parameter space, but signal samples are computationally expensive to produce.
}{signal/truth_lhe_HH_m}
%{signal/truth_lhe_jj_M}
{signal/truth_lhe_HH_dEta}


\newcommand{\lcm}[2]{
    #1_{#2}(
        m_{HH},
        m_{jj},
        ...
    )
}
\newcommand{\lcma}[1]{\lcm{a}{#1}}
\newcommand{\lcmb}[1]{\lcm{b}{#1}}

\frame{
    \frametitle{3D Coupling Dependence}
    {\footnotesize
        Some kind of shortcut is needed,
            but relationship between matrix element expansion and event yields
            is highly non-trivial due to distortion via acceptance/efficiency.

        \begin{equation} \begin{split}
            \sigma \propto \kv^2 \kl^2 a_1 + \kv^4 a_2 + \kvv^2 a_3 + \kv^3 \kl a_4 + \kv \kl \kvv a_5 + \kv^2 \kvv a_6
            \nonumber
        \end{split} \end{equation}
    }
    \begin{center}
    \resizebox{3mm}{!}{$\downarrow$}
    \end{center}
    {\footnotesize \begin{equation} \begin{alignedat}{3}
        & \frac{d^n \sigma}{d m_{HH} d m_{jj} d ...} = && &&\\
        \kv^2 \kl^2 &\times \lcma{1}
            + \qquad\:\: \kv^4 &&\times \lcma{2}
            + \quad\:\: \kvv^2 &&\times \lcma{3} \\
        + \kv^3 \kl &\times \lcma{4}
            + \kv \kl \kvv &&\times \lcma{5}
            + \kv^2 \kvv &&\times \lcma{5}
            \nonumber
    \end{alignedat} \end{equation} }
    \vspace{3mm}

    \begin{center}
    \resizebox{4mm}{!}{$\downarrow$}
    \begin{minipage}[b]{40mm}{\tiny
        hadronization, material interaction,\\
        reconstruction, selection
    }\end{minipage}
    \end{center}


    {\footnotesize \begin{equation} \begin{alignedat}{3}
        & \lcm{N}{\textrm{events}} = && && \\ 
        \kv^2 \kl^2 &\times \lcmb{1}
            + \qquad\:\: \kv^4 &&\times \lcmb{2}
            + \quad\:\: \kvv^2 &&\times \lcmb{3} \\
        + \kv^3 \kl &\times \lcmb{4}
            + \kv \kl \kvv &&\times \lcmb{5}
            + \kv^2 \kvv &&\times \lcmb{5}
            \nonumber
    \end{alignedat} \end{equation} }
}

% 6 variables, six equations, make a mess
\frame{
    \frametitle{6 Unknowns: Solve With 6 Equations}

    \vspace{5mm}

    \foreach \index in {1,2,3,4,5,6}{
        { \small $ N_\index
            = \fkv{\index}^2 \fkl{\index}^2 b_1
            + \fkv{\index}^4 b_2
            + \fkvv{\index}^2 b_3
            + \fkv{\index}^3 \fkl{\index} b_4
            + \fkv{\index} \fkl{\index} \fkvv{\index} b_5
            + \fkv{\index}^2 \fkvv{\index} b_6 $ }

        \vspace{5mm}
    }
}

% convert to matrix, show nice matrix solution
\frame{
    \frametitle{Linear Algebra Makes this Simple}

    \begin{columns}[T]
        \begin{column}{0.18\textwidth}
            $ \vec{N} = \begin{pmatrix} N_1 \\ N_2 \\ N_3 \\ N_4 \\ N_5 \\ N_6 \end{pmatrix} $
        \end{column}
        \begin{column}{0.15\textwidth}
            $ \vec{b} = \begin{pmatrix} b_1 \\ b_2 \\ b_3 \\ b_4 \\ b_5 \\ b_6 \end{pmatrix} $
        \end{column}
        \begin{column}{0.25\textwidth}
            $ \vec{f} = \begin{pmatrix} \kv^2 \kl^2 \\ \kv^4 \\ \kvv^2 \\ \kv^3 \kl \\ \kv \kl \kvv \\ \kv^2 \kvv \end{pmatrix} $
        \end{column}
        \begin{column}{0.4\textwidth}
            $ F = \begin{pmatrix}
                \vec{f}(\fkvv{1}, \fkl{1}, \fkv{1}) \\
                \vec{f}(\fkvv{2}, \fkl{2}, \fkv{2}) \\
                \vec{f}(\fkvv{3}, \fkl{3}, \fkv{3}) \\
                \vec{f}(\fkvv{4}, \fkl{4}, \fkv{4}) \\
                \vec{f}(\fkvv{5}, \fkl{5}, \fkv{5}) \\
                \vec{f}(\fkvv{6}, \fkl{6}, \fkv{6}) \\
            \end{pmatrix} $
        \end{column}
    \end{columns}

    \vspace{10mm}

    $ \vec{N} = F \bullet \vec{b} \; \Longrightarrow \; \vec{b} = F^{-1} \bullet \vec{N} $

    \vspace{10mm}

    $ \boxed{ N(\kvv,\kl,\kv) = \vec{f}(\kvv,\kl,\kv) \bullet F^{-1} \bullet \vec{N} } $
}


\frame{
    \frametitle{Chosen 6-Sample Combination}
    Linear combination equation used to combine basis MC samples together:
    \begin{columns}
        \begin{column}{0.19\textwidth}
            \begin{center} 
            {\tiny Signal Sample Basis Set}

            \resizebox{0.2\textheight}{!}{ \begin{tabular}{ |l|l|l| }
                \hline
                \textbf {$\kappa_{2V}$} & \textbf {$\kappa_\lambda$} & \textbf {$\kappa_V$} \\
                \hline
                    1   &   1 & 1   \\
                    1.5 &   1 & 1   \\
                    1   &   2 & 1   \\
                    1   &  10 & 1   \\
                    1   &   1 & 0.5 \\
                    0   &  -5 & 0.5 \\
                \hline
            \end{tabular}}
            \end{center}

        \end{column}
        \begin{column}{0.8\textwidth}
            \resizebox{0.9\textwidth}{!}{ \begin{minipage}{1.0\textwidth}

            \vspace{10mm}
            {\tiny \begin{equation}
            \begin{split}
                \textrm{MC}'(\kvv, \kl, \kv) = \hspace{50mm}& \\
                \left(\frac{68 \kappa_{2V}^{2}}{135} - 4 \kappa_{2V} \kappa_{V}^{2} + \frac{20 \kappa_{2V} \kappa_{V} \kappa_{\lambda}}{27} + \frac{772 \kappa_{V}^{4}}{135} - \frac{56 \kappa_{V}^{3} \kappa_{\lambda}}{27} + \frac{\kappa_{V}^{2} \kappa_{\lambda}^{2}}{9}\right)
                    \times & \textrm{MC}{\left(1,1,1 \right)} \\
                + \left(- \frac{4 \kappa_{2V}^{2}}{5} + 4 \kappa_{2V} \kappa_{V}^{2} - \frac{16 \kappa_{V}^{4}}{5}\right)
                    \times & \textrm{MC}{\left(\frac{3}{2},1,1 \right)} \\
                + \left(\frac{11 \kappa_{2V}^{2}}{60} + \frac{\kappa_{2V} \kappa_{V}^{2}}{3} - \frac{19 \kappa_{2V} \kappa_{V} \kappa_{\lambda}}{24} - \frac{53 \kappa_{V}^{4}}{30} + \frac{13 \kappa_{V}^{3} \kappa_{\lambda}}{6} - \frac{\kappa_{V}^{2} \kappa_{\lambda}^{2}}{8}\right)
                    \times & \textrm{MC}{\left(1,2,1 \right)} \\
                + \left(- \frac{11 \kappa_{2V}^{2}}{540} + \frac{11 \kappa_{2V} \kappa_{V} \kappa_{\lambda}}{216} + \frac{13 \kappa_{V}^{4}}{270} - \frac{5 \kappa_{V}^{3} \kappa_{\lambda}}{54} + \frac{\kappa_{V}^{2} \kappa_{\lambda}^{2}}{72}\right)
                    \times & \textrm{MC}{\left(1,10,1 \right)}  \\
                + \left(\frac{88 \kappa_{2V}^{2}}{45} - \frac{16 \kappa_{2V} \kappa_{V}^{2}}{3} + \frac{4 \kappa_{2V} \kappa_{V} \kappa_{\lambda}}{9} + \frac{152 \kappa_{V}^{4}}{45} - \frac{4 \kappa_{V}^{3} \kappa_{\lambda}}{9}\right)
                    \times & \textrm{MC}{\left(1,1,\frac{1}{2} \right)} \\
                + \left(\frac{8 \kappa_{2V}^{2}}{45} - \frac{4 \kappa_{2V} \kappa_{V} \kappa_{\lambda}}{9} - \frac{8 \kappa_{V}^{4}}{45} + \frac{4 \kappa_{V}^{3} \kappa_{\lambda}}{9}\right)
                    \times & \textrm{MC}{\left(1,-5,\frac{1}{2} \right)}
                \nonumber
            \end{split} \end{equation}}

            \end{minipage}}
        \end{column}
    \end{columns}
}

\displaythree{Validation of Combination Technique}{
    Strong agreement is found between combination signal and MC-generated signal
}
{signal/reco_mHH_cvv1p00cl1p00cv1p50}
{signal/reco_mHH_cvv1p00cl0p00cv1p00}
{signal/reco_mHH_cvv3p00cl1p00cv1p00}

\displaythree{Checking Off-Axis Regions}{
    Though there are no MC samples to compare to here,
        but visual inspection shows that the combination signal at points
        far from the SM is at least well behaved
}
{signal/preview_reco_mHH_new_cvv2p50cl-10p00cv1p00}
{signal/preview_reco_mHH_new_cvv2p00cl-10p00cv1p00}
{signal/preview_reco_mHH_new_cvv0p00cl13p00cv1p00}

{\displayonelarge{Combination Performance}{
    Chosen basis shows very few regions of negative bins
    \vspace{5mm}

    {\footnotesize cyan dots indicate location of basis samples}


}{signal/negative_weights_base}

