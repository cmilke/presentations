\section{I'll figure this out later I just wanna jot this down for myself first}

    Here's my more formal attempt at writing down what I do in the 2/3D interpolation

    Overall goal is to measure cross-section of an interaction: ``$\Xsec$".

    Cross-section cannot be measured directly. Instead, we must measure the total number of events ``$T$". 

    The total number of events can be related to the cross-section via the integrated luminosity $L$, by the simple relation $T = \Xsec \times L$.

    This calculation assumes that we can measure all particles across all phase space (emitted at all angles). In practice though, this is not the case.
    Many events are very forward and are never detected by the detector elements.
    Furthermore, due to the need to subtract out background events, many kinematic cuts must be placed on events, further reducing the phase space actually available.
    It is useful then to instead look at the number of events with particles existing in a particular region of phase $\mathscr{E} (q) \equiv \xsec (q) \times L$, where ``$\xsec$" is the differential cross-section.

    The total cross-section is related to the differential cross-section by the relation \\
    $\xsec (q) = d \Xsec / d q$, which can be rearanged for cross-section by integrating over phase space $q$: \\
    $\Xsec = \int \xsec (q) d q$.

    Relating this back to the event counts, we have\\
    $\frac{1}{L} T = \int \frac{1}{L} \mathscr{E} (q) dq $ \\
    $T = \int \mathscr{E} (q) dq $.

    In order to bring this in line with the fact that we cannot simulate an infinite number of events, let's decompose the integral into an infinite sum:\\
    $T = \lim\limits_{N\to\infty} \sum\limits_{n=0}^{N} \mathscr{E}(q_n) dq = \lim\limits_{N\to\infty} \sum\limits_{n=0}^{N} \mathscr{E}(q_n) \Delta q / N $.

    This total is equivalent to the theoretical value only if the number of events is infinite (and thus able to represent all of phase space).
    With a finite number of events, the theoretical total is reduced to an approximation $\tau$, where\\
    $\tau = \sum\limits_{n=0}^{N} \mathscr{E}(q_n) \Delta q / N $.

    The relative space any one event takes in phase space is accounted for in monte-carlo by giving events weights, so this can be rewritten as\\
    $\tau = \sum\limits_{n=0}^{N} w(q_n) = \sum\limits_{n=0}^{N} w_n \quad,\quad w(q_n) \equiv \mathscr{E}(q_n) \Delta q / N $.

    Final analysis of events does not count all events, but rather looks at the distribution of event counts as a function of some specific element of phase space.
    In this analysis, that element is the di-Higgs invariant mass \mhh, and so the event count per bin $m$ in \mhh can be written as\\
    $\tau_m = \sum\limits_{n=0}^{N} w_{nm} $. 

    The effects of performing reconstruction and selection can be represented by a multiplicative factor ``$z(q)$" corresponding to the probability an event with phase-space parameters $q$ will survive selection.
    What remains is the reconstructed event yield\\
    $U(q) = \int z(q) \mathscr{E}(q) dq $.

    Which can be returned to the discrete, finite-event case as\\
    $\mu = \sum\limits_{n=0}^{N} z_n * w_{nm} $.

    Note that in the infinite, continuous case, $z(q)$ is purely binary (in continuous phase-space an event either passes selection or not),
        but becomes a probability $z_n$ when regions of phase-space are aggregated together into the same bin.


\section{Sample Interpolation}

    Simulating, reconstructing, and performing selection on stuff is hard.
    We need to check all points in $\kappa$ space though.
    To address this we can exploit the underlying field-theory mechanics to reverse-engineer a general equation for the number of events expected for any value of the $\kappa$ couplings.
    First, the influence of the couplings can be related to the cross-section through the interaction amplitude ``$\amp$", where\\
    $\amp(q,\kvv,\kl,\kv) =  \kv \kl \matel_s(q) + \kv^2 \matel_t(q) + \kvv \matel_X(q) $

    The differential cross-section is just the absolute square of the amplitude,\\
    \begin{equation}\begin{split}
        \xsec(q,\kvv,\kl,\kv) = |\amp(q,\kvv,\kl,\kv)|^2 &= 
          \kv^2 \kl^2 \matel_s^2(q) + \kv^4 \matel_t^2(q) + \kvv^2 \matel_X^2(q) \\
        &+ \kv^3 \kl (\matel_s^*(q) \matel_t(q) + \matel_t^*(q) \matel_s(q)) \\
        &+ \kv \kl \kvv (\matel_s^*(q) \matel_X(q) + \matel_X^*(q) \matel_s(q) ) \\
        &+ \kv^2 \kvv (\matel_t^*(q) \matel_X(q) + \matel_X^*(q) \matel_t(q) )
    \end{split} \end{equation}

    And the abundance of matrix element cross terms can be absorbed into simple coefficients $a_i(q)$\\
    $\xsec(q,\kvv,\kl,\kv) = \kv^2 \kl^2 a_1(q) + \kv^4 a_2(q) + \kvv^2 a_3(q) + \kv^3 \kl a_4(q) + \kv \kl \kvv a_5(q) + \kv^2 \kvv a_6(q) $


    This can be further simplified by using just $\kappa$ as shorthand for all the couplings,
    so  $\xsec(q,\kvv,\kl,\kv) \to  \xsec(q,\kappa) $, and by collecting the various couplings into a vector\\
    $ \vec{f}(\kappa) = \begin{pmatrix} \kv^2 \kl^2 \\ \kv^4 \\ \kvv^2 \\ \kv^3 \kl \\ \kv \kl \kvv \\ \kv^2 \kvv \end{pmatrix} $

    So we now have\\
    $\xsec(q,\kappa) = f_1(\kappa) a_1(q) + f_2(\kappa) a_2(q) + f_3(\kappa) a_3(q) + f_4(\kappa) a_4(q) + f_5(\kappa) a_5(q) + f_6(\kappa) a_6(q) $

    Which can be written as:  
    $\xsec(q,\kappa) = \vec{a}(q) \bullet \vec{f}(\kappa) $.

    or, adopting Einstein notation, as
    $\xsec(q,\kappa) = a_i(q) f_i(\kappa) $.

    From here we can write this in terms of observed events\\ 
    $T(\kappa) = \int \xsec(q,\kappa) dq \times L = \int a_i(q) f_i(\kappa) dq \times L $.

    And in terms of post-selection events as\\
    $U(\kappa) = \int z(q) \xsec(q,\kappa) dq \times L = \int z(q) a_i(q) f_i(\kappa) dq \times L $.

    To get the number of expected events for any values of the couplings, we only need to find the reconstructed forms of $a(q)_{i}$.
    By running simulations for 6 different, linearly independent variations of $\xsec$, we can obtain the six equations needed to solve for six variables:\\
    $\xsec(q, \kappa_j) = a(q)_{i} f_i(\kappa_j) $, for $j \in {1-6}$. \\
    $\xsec(q, \kappa_j) \to  \xsec_j(q) $, $f_i(\kappa_j) \to F_{ij} $, \\
    $\xsec_{j}(q) = a(q)_{i} F_{ij}$

    Solving for $a$ is just a matter of inverting the matrix $F_{ij}$ \\
    $\xsec_{j}(q) \times F_{ij}^{-1}= a(q)_{i} F_{ij} \times F_{ij}^{-1}$ \\
    $a_{i}(q) = \xsec_{j}(q) F_{ij}^{-1}$ \\
    $a_{i}(q) = \xsec_{j}(q) G_{ji}$, with $G_{ji} \equiv F_{ij}^{-1}$ \\

    With the primed ``$\xsec'$" now denoting a linearly combined cross-section,
        the cross-section for an arbitrary value of the couplings can then be written as
    \begin{equation} \begin{split}
        \xsec'(q, \kappa) &= a_i(q) f_i(\kappa) \\
        \xsec'(q, \kappa) &= \xsec_{j}(q) G_{ji} f_i(\kappa) \\
        \xsec'(q, \kappa) &=  G_{ji} f_i(\kappa) \xsec_{j}(q) \\
        \xsec'(q, \kappa) &=  g_j(\kappa) \xsec_{j}(q) \quad, g_j(\kappa) \equiv G_{ji} f_i(\kappa)
    \end{split} \end{equation}

    Rewriting this in terms of events:
    \begin{equation} \begin{split}
        T'(\kappa) &= \int \xsec'(q,\kappa) dq \times L \\
        T'(\kappa) &= \boxed{\int g_j(\kappa) \xsec_{j}(q) dq \times L = g_j(\kappa) \int \xsec_{j}(q) dq \times L} \\
        T'(\kappa) &= g_j(\kappa) \Xsec_j \times L = g_j(\kappa) T_j \\
        T'(\kappa) &= g_j(\kappa) T_j
    \end{split} \end{equation}

    Note: the boxed section is the key to why this whole combination system works at the event-level as we use it.
    The core reason this can be done is that the $g_j(\kappa)$ coefficient
        \textit{is independant of phase space},
        and thus can be pulled out of the phase space integral.


    This can be repeated for post-selection phase-space:
    \begin{equation} \begin{split}
        U'(\kappa) &= \int z(q) \xsec'(q,\kappa) dq \times L = \int z(q) g_j(\kappa) \xsec_{j}(q) dq \times L \\
        U'(\kappa) &= g_j(\kappa) \int z(q) \xsec_{j}(q) dq \times L = g_j(\kappa) U_j
    \end{split} \end{equation}

    And performed approximately for the discrete, finite case:
    \begin{equation} \begin{split}
        U'(\kappa) &= g_j(\kappa) \int z(q) \xsec_{j}(q) dq \times L \\
        U'(\kappa) &= g_j(\kappa) \lim\limits_{N\to\infty} \sum\limits_{n=0}^{N} z(q) \xsec_{j}(q) dq \times L \\
        U'(\kappa) &= g_j(\kappa) \lim\limits_{N\to\infty} \sum\limits_{n=0}^{N} z(q_n) \xsec_{j}(q_n) \Delta q_n / N \times L \\
        U'(\kappa) \approx \mu' &= g_j(\kappa) \sum\limits_{n=0}^{N} z_n \xsec_{j,n} \Delta q_n / N \times L \\
        \mu' &= g_j(\kappa) \sum\limits_{n=0}^{N} z_n w_{j,n} \times L \\
        \mu' &= g_j(\kappa) \mu_j
    \end{split} \end{equation}



    %Since an observed event count is a collection of many individual events, this can be re-written in terms of individual event weights:\\
    %$T_m(\kappa) = G_{ji} f_i(\kappa) T_{mj}$,\\
    %$T_m(\kappa) = G_{ji} f_i(\kappa) z_{mj} \sum\limits_{n=0}^{N} w_{nmj} $. FIXME I'm not so sure about this way of describing reco-level events







