
\frame{
    \frametitle{Variation Scan Range}

    Wide range of values to check, brute-force approach:
    \vspace{3mm}

    \begin{itemize} {
        \item $\kvv =$ numpy.arange(-1, 4.5, 0.5)
        \item $\kl =$ numpy.arange(-9, 11, 1)
        \item $\kv =$ [ 0.5, 1, 1.5 ]
        \item 660 different variations
        \item Require all combinations to include SM and $\kl=2$ point to reduce combinatorics
        \item Each variation then has $\approx$ 227 valid possible combinations
        \item Total of $\approx$ 148,820 different combinations
    } \end{itemize}
    \vspace{3mm}

    Add each new variation to available samples one at a time,
    recalculate the solidarity integral for all possible valid combinations with the new sample in place
}

\displaytwocaption{Solidarity Histograms}{
    Each variation tested produces $\approx$ 227 combinations,
    and each combination has its own solidarity integral.
}
{projective_solidarity_all}{Overview of all 660 variations (sorted by Sum-Total of all integrals)}
{projective_solidarity_dump}{Closer look at top 10 variations}

\displaytwo{Performance Predictions}{
    Sum together the solidarity integrals of all possible combinations to produce a heatmap of potential improvement.
    \vspace{3mm}

    Improvement largely seems to come from placing a sample in the middle of the poor-performing regions
}
{negative_weights_toprank0}
{solidarity_performance_kv1}

\displaythree{Performance Predictions with Different $\kv$}{
    Variations with non-SM $\kv$ values can also potentially help
}
{solidarity_performance_kv0.5}
{solidarity_performance_kv1}
{solidarity_performance_kv1.5}



%\displayfour{Post-Selection Performance (Top) VS Truth-Level (Bottom)}
%{negative_weights_toprank0}
%{negative_weights_toprank1}
%{negative_weightsuncappedtruthRtop0}
%{negative_weightsuncappedtruthRtop1}
%
%\displayonelarge{Post-Selection/Truth-Level Correlation}{
%    At first glance, truth performance appears only loosely correlated to post-selection performance.
%    \vspace{4mm}
%
%    But, performance can be made more similar by capping the statistics of the truth samples
%}{Nweight_integral_VS_Nweight_uncapped_truth_integral}
%
%
%\displayfour{Post-Selection Performance (Top) VS ``Capped" Truth-Level (Bottom)}
%{negative_weights_toprank0}
%{negative_weights_toprank1}
%{negative_weightstruthRtop0}
%{negative_weightstruthRtop1}
%
%
%\displayonelarge{Post-Selection/Capped-Truth-Level Correlation}{
%    Artificially limiting truth samples to 10\% of their events provides a much stronger correlation between truth and post-selection performance
%}{Nweight_integral_VS_Nweight_truth_integral}
%
