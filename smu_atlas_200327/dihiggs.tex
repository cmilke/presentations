\section{ Di-Higgs Analysis }
\fullscreenimage{VBF Di-Higgs Analysis}{vbf-hh_diagrams}
\frame{
    \frametitle{ VBF $\rightarrow$ HH $\rightarrow$ \fourB }

    \begin{columns}
        \begin{column}{0.5\textwidth}
            { \small
                I've joined the VBF to 4b dihiggs subgroup, to look into improving selection of VBF jets in the final state.
                Ultimate goal would be putting a better limit on the $C_{2V}$ coupling constant.
            }

            \begin{figure}
                \includegraphics[width=\linewidth,height=\textheight,keepaspectratio]
                {dihiggs_task_list}
            \end{figure}
        \end{column}
        \begin{column}{0.5\textwidth}
            \resizebox{0.50\textheight}{!}{
                \begin{tikzpicture} \begin{feynman}
    \vertex (c2v) {\tiny $C_{2V}$};
    \vertex [above right=of c2v] (h1) {H};
    \vertex [below right=of c2v] (h2) {H};
    \vertex [above left=of c2v] (vb1);
    \vertex [below left=of c2v] (vb2);
    \vertex [left=of vb1] (q1i) {q};
    \vertex [left=of vb2] (q2i) {q};
    \vertex [above right=of h1] (b1) {$b$};
    \vertex [below right=of h2] (bbar2) {$\bar b$};
    \vertex [below=of b1] (bbar1) {$\bar b$};
    \vertex [above=of bbar2] (b2) {$b$};

    \vertex [above=of b1] (q1f) {q};
    \vertex [below=of bbar2] (q2f) {q};

    \diagram* {
        (q1i) -- (vb1) -- (q1f),
        (q2i) -- (vb2) -- (q2f), 
        (vb1) -- [boson] (c2v) -- [boson] (vb2),
        (h1) -- [scalar] (c2v) -- [scalar] (h2),
        (b1) -- (h1) -- (bbar1),
        (b2) -- (h2) -- (bbar2),
    };
\end{feynman} \end{tikzpicture}

            }
        \end{column}
    \end{columns}
}

\frame{
    \frametitle{ Immediate Task and Further Objectives }
    \begin{itemize} {
        \item Immediate task is to check (and possibly try to improve) efficiency of current algorithm in selecting which 2 jets are the VBF jets
        \item Next steps would be using ML and Jet Substructure algorithms to improve VBF/ggF discrimination efficiency
        \begin{itemize} {
            \item Centrality
            \item Jet Pull
            \item Jet Charge
        } \end{itemize}
        \item Further studies involve testing discrimination efficiency with non-SM values of \cvv
        \item Much more to come on this soon...
    } \end{itemize}
}


% Initial Task: how often are the higgs pairs/vbf pairs correctly identified?
% Current approach is a BDT to identify the two jet pairs that correspond to the higgss,
% and then use a number of heuristic cuts to choose vbf jets* (I think...)
% Using ML and jet substructure tools, can I do better?
