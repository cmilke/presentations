\documentclass{beamer}
%Imports and customization
\usepackage{tikz}
\usepackage{graphicx}
\usepackage{tikz-feynman}
\graphicspath{ 
    {./images/}
}

\beamertemplatenavigationsymbolsempty
\setbeamertemplate{sidebar right}{}
\setbeamertemplate{footline}{
    \hfill\usebeamertemplate***{navigation symbols}
    \hspace{1cm}\insertframenumber{}/\inserttotalframenumber
}
\setbeamertemplate{caption}{\raggedright\insertcaption\par}
\setbeamersize{text margin left=5mm,text margin right=5mm} 

\setbeamerfont{itemize/enumerate body}{size=\scriptsize}
\setbeamerfont{itemize/enumerate subbody}{size=\scriptsize}
\setbeamerfont{itemize/enumerate subsubbody}{size=\scriptsize}


%Custom Macros
\newcommand{\statwarn}{
    \tiny \color{red} Absolute numbers here mean NOTHING. Plots are based on small (100k events) samples, and are highly biased. All that matters is relative position!
}


\newcommand{\fullscreenimage}[2]{
    \frame{
        \frametitle{#1} 
        \begin{figure}
        \includegraphics[height=0.9\textheight,keepaspectratio]{#2}
        \end{figure}
    }
}


\newcommand{\displayone}[3]{
    \frame{
        \frametitle{#1} 
        \begin{columns}
            \begin{column}{0.5\textwidth}
                #2
            \end{column}
            \begin{column}{0.5\textwidth}
                \begin{figure}
                    \includegraphics[width=\linewidth,height=\textheight,keepaspectratio]{#3}
                \end{figure}
            \end{column}
        \end{columns}
    }
}

\newcommand{\displayonelarge}[3]{
    \frame{
        \frametitle{#1} 
        \begin{columns}
            \begin{column}{0.3\textwidth}
                #2
            \end{column}
            \begin{column}{0.7\textwidth}
                \begin{figure}
                    \includegraphics[width=\linewidth,height=\textheight,keepaspectratio]{#3}
                \end{figure}
            \end{column}
        \end{columns}
    }
}


\newcommand{\displaytwo}[4]{
    \frame{
        \frametitle{#1} 
        #2
        \begin{columns}
            \begin{column}{0.5\textwidth}
                \begin{figure}
                    \includegraphics[width=\linewidth,height=\textheight,keepaspectratio]{#3}
                \end{figure}
            \end{column}
            \begin{column}{0.5\textwidth}
                \begin{figure}
                    \includegraphics[width=\linewidth,height=\textheight,keepaspectratio]{#4}
                \end{figure}
            \end{column}
        \end{columns}
    }
}


\newcommand{\displaythree}[5]{
    \frame{
        \begin{columns}[T]
            \begin{column}{0.5\textwidth}
                \insertframetitle{#1}\\
                #2
            \end{column}
            \begin{column}{0.5\textwidth}
                \begin{figure}
                    \includegraphics[width=\linewidth,height=\textheight,keepaspectratio]{#3}
                \end{figure}
            \end{column}
        \end{columns}
        \begin{columns}[T]
            \begin{column}{0.5\textwidth}
                \begin{figure}
                    \includegraphics[width=\linewidth,height=\textheight,keepaspectratio]{#4}
                \end{figure}
            \end{column}
            \begin{column}{0.5\textwidth}
                \begin{figure}
                    \includegraphics[width=\linewidth,height=\textheight,keepaspectratio]{#5}
                \end{figure}
            \end{column}
        \end{columns}
    }
}


\newcommand{\announcesection}[1]{
    \section{#1}
    \frame{
        \begin{center}
            {\huge #1} 
        \end{center}
    }
}


%Begin Presentation
\begin{document}
    \title{Investigating Possible Use of Photon Radiation in VBF Di-Higgs Selection}
    \author{Chris Milke}
    \date{26 August, 2020}

    \frame{\titlepage}
    %\frame{\frametitle{Overview} \tableofcontents}

    \displayonelarge{Photon Radiation in VBF}
        {\small Based on {\color{blue}\hyperlink{https://atlas.web.cern.ch/Atlas/GROUPS/PHYSICS/CONFNOTES/ATLAS-CONF-2020-044/}{ATLAS CONF Note 2020-44}},
            I'm investigating the use of high-$E_T$ radiated photons as a way to isolate VBF events from background.
        }
        {photon_leading}

    \displaythree{Photon $E_T$ Distribution For \cvv 0 and 1}
        {
            While \cvv=0 shows nearly double the incidence of high-$E_T$ photons,
            both cases present a significant number of events with such photons.
        }
        {truth_photon_Et_genmerge}
        {truth_photon_Et_genmerge_zoom0}
        {truth_photon_Et_genmerge_sumnorm}

    \frame{
        \frametitle{Feasibility of Using Photons for Alternate Trigger}
        \begin{tabular}{ |l|l|l|l|l| }
            \hline
            \textbf{\cvv,} & \textbf{\% Pass} & \textbf{\% w/ $\gamma$} & \textbf{\% w/ $\gamma$} & \textbf{\% w/ $\gamma$} \\
            \textbf{ MC16a/d/e} & \textbf{Triggers} & \textbf{($E_T>25$GeV)} & \textbf{\& Pass} & \textbf{\& Fail} \\
            \hline
            \hline
            \cvv= 0, MC16a & 63.6 & 43.2 & 32.5 & 10.8 \\
            \hline
            \cvv= 0, MC16d & 60.9 & 43.0 & 33.4 &  9.6 \\
            \hline
            \cvv= 0, MC16e & 66.9 & 43.1 & 35.3 &  7.8 \\
            \hline
            \cvv= 1, MC16a & 44.4 & 22.6 & 13.9 &  8.7 \\
            \hline
            \cvv= 1, MC16d & 36.3 & 22.6 & 12.7 &  9.9 \\
            \hline
            \cvv= 1, MC16e & 45.6 & 22.4 & 14.4 &  8.0 \\
            \hline
        \end{tabular}

        \vspace{10mm}

        {\small For both \cvv=0 \& 1, there are a significant fraction of events with high-$E_T$ photons which do not pass the triggers currently used for VBF-4b.
        A new signal category using a photon-based trigger (and possibly looser trigger cuts elsewhere) could provide more events or a more pure signal sample.}
    }
    
    \announcesection{Backup}

    \displaythree{Photon $E_T$ Distribution For \cvv 0 and 1}
        {Little variation between MC16 generations}
        {truth_photon_Et}
        {truth_photon_Et_zoom0}
        {truth_photon_Et_sumnorm}

    \frame{
        \frametitle{Trigger Pass Rates for 30 GeV Photons}
        \begin{tabular}{ |l|l|l|l|l| }
            \hline
            \textbf{\cvv,} & \textbf{\% Pass} & \textbf{\% w/ $\gamma$} & \textbf{\% w/ $\gamma$} & \textbf{\% w/ $\gamma$} \\
            \textbf{ MC16a/d/e} & \textbf{Triggers} & \textbf{($E_T>30$GeV)} & \textbf{\& Pass} & \textbf{\& Fail} \\
            \hline
            \hline
             \cvv= 0, MC16a & 63.6 & 34.3 & 25.9 & 8.4 \\
             \hline
             \cvv= 0, MC16d & 60.9 & 34.1 & 27.2 & 6.9 \\
             \hline
             \cvv= 0, MC16e & 66.9 & 34.1 & 28.5 & 5.6 \\
             \hline
             \cvv= 1, MC16a & 44.4 & 16.0 & 10.1 & 5.9 \\
             \hline
             \cvv= 1, MC16d & 36.3 & 16.0 &  9.5 & 6.5 \\
             \hline
             \cvv= 1, MC16e & 45.6 & 15.8 & 10.6 & 5.2 \\
            \hline
        \end{tabular}
    }
\end{document}
