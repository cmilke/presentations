\section{BDT Evaluation}

\displaytwo{Naive Cutflow Performance}{
    Using same ggF rejection level (BDT cut=0.05) as the default selection method leads to unexpected boost in ggF acceptance downstream. 
}
{cutflows/cutflow_4tag_old}
{cutflows/cutflow_4tag_new_bad}


\displayonelarge{Comparison of Final Throughput for Various \cvv Values }{
    {\small Depending on the cut, the VBF BDT is able to drastically reduce ggF contamination
    with a limited effect on the VBF samples.}
}{Xwt_plots/final_weight_comparison_FourTagCutflow}

\displayonelarge{Final Throughput ``ROC Curves''}{
    {\small
        The VBF BDT is able to at least match the current cut-based selection process for all studied values of \cvv.
        ggF contamination can be reduced rapidly with tighter values of the BDT cut.
    }
}{Xwt_plots/final_roc_FourTagCutflow}

\displaytwo{Cutflow Performance at Higher Rejection}{
    Using a BDT cut at 10\% efficiency level (BDT cut=0.3) initially causes VBF efficiency to plummet, but later selection cuts make up for this, while ggF acceptance remains low.
}
{cutflows/cutflow_4tag_old}
{cutflows/cutflow_4tag_new}
