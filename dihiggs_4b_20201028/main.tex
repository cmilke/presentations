\documentclass{beamer}
\usepackage{pdfpages}
%Imports and customization
\usepackage{tikz}
\usepackage{graphicx}
\usepackage{tikz-feynman}
\graphicspath{ 
    {./images/}
}

\beamertemplatenavigationsymbolsempty
\setbeamertemplate{sidebar right}{}
\setbeamertemplate{footline}{
    \hfill\usebeamertemplate***{navigation symbols}
    \hspace{1cm}\insertframenumber{}/\inserttotalframenumber
}
\setbeamertemplate{caption}{\raggedright\insertcaption\par}
\setbeamersize{text margin left=5mm,text margin right=5mm} 

\setbeamerfont{itemize/enumerate body}{size=\scriptsize}
\setbeamerfont{itemize/enumerate subbody}{size=\scriptsize}
\setbeamerfont{itemize/enumerate subsubbody}{size=\scriptsize}


%Custom Macros
\newcommand{\statwarn}{
    \tiny \color{red} Absolute numbers here mean NOTHING. Plots are based on small (100k events) samples, and are highly biased. All that matters is relative position!
}


\newcommand{\fullscreenimage}[2]{
    \frame{
        \frametitle{#1} 
        \begin{figure}
        \includegraphics[height=0.9\textheight,keepaspectratio]{#2}
        \end{figure}
    }
}


\newcommand{\displayone}[3]{
    \frame{
        \frametitle{#1} 
        \begin{columns}
            \begin{column}{0.5\textwidth}
                #2
            \end{column}
            \begin{column}{0.5\textwidth}
                \begin{figure}
                    \includegraphics[width=\linewidth,height=\textheight,keepaspectratio]{#3}
                \end{figure}
            \end{column}
        \end{columns}
    }
}

\newcommand{\displayonelarge}[3]{
    \frame{
        \frametitle{#1} 
        \begin{columns}
            \begin{column}{0.3\textwidth}
                #2
            \end{column}
            \begin{column}{0.7\textwidth}
                \begin{figure}
                    \includegraphics[width=\linewidth,height=\textheight,keepaspectratio]{#3}
                \end{figure}
            \end{column}
        \end{columns}
    }
}


\newcommand{\displaytwo}[4]{
    \frame{
        \frametitle{#1} 
        #2
        \begin{columns}
            \begin{column}{0.5\textwidth}
                \begin{figure}
                    \includegraphics[width=\linewidth,height=\textheight,keepaspectratio]{#3}
                \end{figure}
            \end{column}
            \begin{column}{0.5\textwidth}
                \begin{figure}
                    \includegraphics[width=\linewidth,height=\textheight,keepaspectratio]{#4}
                \end{figure}
            \end{column}
        \end{columns}
    }
}


\newcommand{\displaythree}[5]{
    \frame{
        \begin{columns}[T]
            \begin{column}{0.5\textwidth}
                \insertframetitle{#1}\\
                #2
            \end{column}
            \begin{column}{0.5\textwidth}
                \begin{figure}
                    \includegraphics[width=\linewidth,height=\textheight,keepaspectratio]{#3}
                \end{figure}
            \end{column}
        \end{columns}
        \begin{columns}[T]
            \begin{column}{0.5\textwidth}
                \begin{figure}
                    \includegraphics[width=\linewidth,height=\textheight,keepaspectratio]{#4}
                \end{figure}
            \end{column}
            \begin{column}{0.5\textwidth}
                \begin{figure}
                    \includegraphics[width=\linewidth,height=\textheight,keepaspectratio]{#5}
                \end{figure}
            \end{column}
        \end{columns}
    }
}


\newcommand{\announcesection}[1]{
    \section{#1}
    \frame{
        \begin{center}
            {\huge #1} 
        \end{center}
    }
}


%Begin Presentation
\begin{document}
    \setbeamercolor{background canvas}{bg=}
    \title{Exploring 3D Coupling Scan Limits}
    \author{Chris Milke}
    \date{29 October, 2020}

    \frame{\titlepage}
    \frame{\frametitle{Overview} \tableofcontents}

    \section{Linear Algebra}

% show diagrams with terms, then show squaring
\frame{
    \frametitle{3D Coupling Dependence}
    \begin{figure}
    \includegraphics[width=0.6\linewidth,height=0.6\textheight,keepaspectratio]{vbf-hh_diagrams2b}
    \end{figure}
    $ \sigma = | \kv \kl M_s + \kv^2 M_t + \kvv M_x |^2 = $

    \vspace{10mm}

    $ \kv^2 \kl^2 M_s^2 + \kv^4 M_t^2 + \kvv^2 M_x^2 
    + \kv^3 \kl (M_s^* M_t + M_t^* M_s) 
    + \kv \kl \kvv (M_s^* M_x + M_x^* M_s ) 
    + \kv^2 \kvv (M_t^* M_x + M_x^* M_t )$

    \vspace{10mm}

    $ \sigma = \kv^2 \kl^2 a_1 + \kv^4 a_2 + \kvv^2 a_3 + \kv^3 \kl a_4 + \kv \kl \kvv a_5 + \kv^2 \kvv a_6 $
}

    \section{Orthogonal Basis States}

\frame{
    \frametitle{Initial Attempt at a Basis Set}
    \begin{columns}
        \begin{column}{0.4\textwidth}
            \begin{center} 
            Basis Set
            \resizebox{0.4\textheight}{!}{\begin{tabular}{ |l|l|l| }
                \hline
                \textbf {$\kappa_{2V}$} & \textbf {$\kappa_\lambda$} & \textbf {$\kappa_V$} \\
                \hline
                1   &  1 &   1 \\
                1   &  0 &  -1 \\
                0   &  1 &   1 \\
                1.5 &  1 &   1 \\
                1   &  2 &   1 \\
                2   &  1 &  -1 \\
                \hline
            \end{tabular}} \end{center}
        \end{column}
        \begin{column}{0.6\textwidth}
            Linear Combination Equation
            \vspace{10mm}

            {\tiny \input{final_amplitude.tex}}
        \end{column}
    \end{columns}
}

\displaythree{Comparing Linear Combination with Generated Sample}{
    {\tiny The linear combination successfully fits separately generated MC samples}
}{mHH_cvv0p5cl1p0cv1p0}
{mHH_cvv2p0cl1p0cv1p0}
{mHH_cvv0p0cl0p0cv1p0}

\displaythree{More plots}{
    {\small The $\kappa_{2V}=1, \kappa_\lambda=10, \kappa_V=1$ state shows large discrepencies, indicating room for improvement.}

    \vspace{5mm}


    { \tiny Note that the basis $\kappa_{2V}=1,\kappa_\lambda=0,\kappa_V=-1$ is one of the basis states. }

}{mHH_cvv0p0cl0p0cv1p0}
{mHH_cvv1p0cl0p0cv-1p0}
{mHH_cvv1p0cl10p0cv1p0}


\fullscreenimage{Normalizing to max-per-bin highlights where states excel}{coupling_scan_rnd_hash_max}



\frame{
    \frametitle{Optimal Basis States for New Production }
    \begin{columns}
        \begin{column}{0.33\textwidth}
            \begin{center} 
            \resizebox{0.3\textheight}{!}{\begin{tabular}{ |l|l|l| }
                \hline
                \textbf {$\kappa_{2V}$} & \textbf {$\kappa_\lambda$} & \textbf {$\kappa_V$} \\
                \hline
                 1.  &  1. &  1.  \\
                 1.5 &  1. &  1.  \\
                 2.  &  1. &  1.  \\
                 1.  &  0. &  1.  \\
                 1.  & 10. &  1.  \\
                 1.  &  1. &  0.5 \\
                \hline
            \end{tabular}}

            \begin{figure}
                \includegraphics[width=\linewidth,height=\textheight,keepaspectratio]
                {coupling_scan_auto_chosenR0_hash_max}
            \end{figure}

            \end{center}
        \end{column}
        \begin{column}{0.33\textwidth}
            \begin{center} 
            \resizebox{0.3\textheight}{!}{\begin{tabular}{ |l|l|l| }
                \hline
                \textbf {$\kappa_{2V}$} & \textbf {$\kappa_\lambda$} & \textbf {$\kappa_V$} \\
                \hline
                 1.  &  1. &  1.  \\
                 1.5 &  1. &  1.  \\
                 2.  &  1. &  1.  \\
                 1.  &  0. &  1.  \\
                 1.  & 10. &  1.  \\
                 1.  &  1. &  1.5 \\
                \hline
            \end{tabular}}

            \begin{figure}
                \includegraphics[width=\linewidth,height=\textheight,keepaspectratio]
                {coupling_scan_auto_chosenR1_hash_max}
            \end{figure}

            \end{center}
        \end{column}
        \begin{column}{0.33\textwidth}
            \begin{center} 
            \resizebox{0.3\textheight}{!}{\begin{tabular}{ |l|l|l| }
                \hline
                \textbf {$\kappa_{2V}$} & \textbf {$\kappa_\lambda$} & \textbf {$\kappa_V$} \\
                \hline
                 1.  &  1. &  1.  \\
                 1.5 &  1. &  1.  \\
                 2.  &  1. &  1.  \\
                 1.  &  0. &  1.  \\
                 1.  & 10. &  1.  \\
                 0.  &  0. &  1.  \\
                \hline
            \end{tabular}}

            \begin{figure}
                \includegraphics[width=\linewidth,height=\textheight,keepaspectratio]
                {coupling_scan_auto_chosenR2_hash_max}
            \end{figure}

            \end{center}
        \end{column}
    \end{columns}
}

    \section{Truth Reweighting}

\frame{
    \frametitle{Attempting Truth Reweighting}
    Use the six truth basis samples to reweight standard model NNT,
    via reweight function:

    \vspace{5mm}

    $ w(\kvv=x,\kl=y,\kv=z, \textrm{bin}\,i) 
     = \frac{ m_{HH}^{\kvv=1,\kl=1,\kv=1,i}}{m_{HH}^{\kvv=x,\kl=y,\kv=z,i} }$

}
\displaytwocaption{Truth Reweighting Results}
{The results are not inspiring...}
{c2v_scan_scan_test001_samps_vbf_pd_161718_xsec}{Current 1D Reweighting}
{c2v_scan_scan_test002_samps_vbf_pd_161718_xsec}{Truth Reweighting}


\displaythree{Truth Reweighting Has Obstacles}
{   
    Truth reweighting places such poor limits because 
    full reconstruction and selection heavily sculpt $m_{HH}$ distributions
}
{reco_mHH_cvv1p0cl1p0cv1p0}
{reco_mHH_cvv2p0cl1p0cv1p0}
{reco_mHH_cvv0p5cl1p0cv1p0}

    \section{NNT Linear Combinations}


\frame{
    \frametitle{Why Didn't I Reweight Directly With NNTs?}
    \begin{columns}
        \begin{column}{0.5\textwidth}
            Production JIRA lists 13 variations for current MC
        \end{column}
        \begin{column}{0.5\textwidth}
            \begin{center} 
            Current MC Samples 
            \resizebox{0.5\textheight}{!}{\begin{tabular}{ |l|l|l|l| }
                \hline
                \textbf {DSID} & \textbf {$\kappa_{2V}$} & \textbf {$\kappa_\lambda$} & \textbf {$\kappa_V$} \\
                \hline
                    450044  &   1   & 1   & 1   \\ 
                    450045  &   1   & 2   & 1   \\ 
                    450046  &   2   & 1   & 1   \\ 
                    450047  &   1.5 & 1   & 1   \\ 
                    450048  &   1   & 1   & 0.5 \\ % !!
                    450049  &   0.5 & 1   & 1   \\ 
                    450050  &   0   & 1   & 1   \\ 
                    450051  &   0   & 1   & 0.5 \\ % !!
                    450052  &   1   & 0   & 1   \\ 
                    450053  &   0   & 0   & 1   \\ % !!
                    450054  &   4   & 1   & 1   \\ 
                    450055  &   1   & 10  & 1   \\ 
                    450056  &   1   & 1   & 1.5 \\ % !!
                \hline
            \end{tabular}} \end{center}
        \end{column}
    \end{columns}
}

\frame{
    \frametitle{Why Didn't I Reweight Directly With NNTs?}
    \begin{columns}
        \begin{column}{0.5\textwidth}
            Production JIRA lists 13 variations for current MC

            \vspace{5mm}

            Only 9 of these were put to use. The DAODs of the other 4 are outdated.
        \end{column}
        \begin{column}{0.5\textwidth}
            \begin{center} 
            Current MC Samples 
            \resizebox{0.5\textheight}{!}{\begin{tabular}{ |l|l|l|l| }
                \hline
                \textbf {DSID} & \textbf {$\kappa_{2V}$} & \textbf {$\kappa_\lambda$} & \textbf {$\kappa_V$} \\
                \hline
                    450044  &   1   & 1   & 1   \\ 
                    450045  &   1   & 2   & 1   \\ 
                    450046  &   2   & 1   & 1   \\ 
                    450047  &   1.5 & 1   & 1   \\ 
     \rowcolor{red} 450048  &   1   & 1   & 0.5 \\ % !!
                    450049  &   0.5 & 1   & 1   \\ 
                    450050  &   0   & 1   & 1   \\ 
     \rowcolor{red} 450051  &   0   & 1   & 0.5 \\ % !!
                    450052  &   1   & 0   & 1   \\ 
     \rowcolor{red} 450053  &   0   & 0   & 1   \\ % !!
                    450054  &   4   & 1   & 1   \\ 
                    450055  &   1   & 10  & 1   \\ 
     \rowcolor{red} 450056  &   1   & 1   & 1.5 \\ % !!
                \hline
            \end{tabular}} \end{center}
        \end{column}
    \end{columns}
}

\frame{
    \frametitle{Why Didn't I Reweight Directly With NNTs?}
    \begin{columns}
        \begin{column}{0.5\textwidth}
            Production JIRA lists 13 variations for current MC

            \vspace{5mm}

            Only 9 of these were put to use. The DAODs of the other 4 are outdated.

            \vspace{5mm}

            There are 84 possible ways of combining 6 of these 9 samples.

            Not a single one of these combinations is valid
        \end{column}
        \begin{column}{0.5\textwidth}
            \begin{center} 
            Current MC Samples 
        \end{column}
    \end{columns}
}

\frame{
    \frametitle{Thanks to Max, a New Variation Will Soon Be Available}
    \begin{columns}
        \begin{column}{0.4\textwidth}
            \begin{center} 
            {\tiny NNT Basis Set}

            \resizebox{0.2\textheight}{!}{ \begin{tabular}{ |l|l|l| }
                \hline
                \textbf {$\kappa_{2V}$} & \textbf {$\kappa_\lambda$} & \textbf {$\kappa_V$} \\
                \hline
                    1.  &   1. & 1.  \\
                    2.  &   1. & 1.  \\
                    1.5 &   1. & 1.  \\
   \rowcolor{green} 0.  &   1. & 0.5 \\
                    1.  &   0. & 1.  \\
                    1.  &  10. & 1.  \\
                \hline
            \end{tabular}}
            \end{center}

        \includegraphics[width=\linewidth,height=\textheight,keepaspectratio]
            {coupling_scan_auto_chosen_reco_R0_hash_max}

        \end{column}
        \begin{column}{0.6\textwidth}
            Linear Combination Equation
            \vspace{10mm}

            {\tiny $
A_{0} \left(2 \kappa_{2V}^{2} - \frac{124 \kappa_{2V} \kappa_{V}^{2}}{9} + \frac{61 \kappa_{2V} \kappa_{V} \kappa_{\lambda}}{9} + \frac{106 \kappa_{V}^{4}}{9} - \frac{17 \kappa_{V}^{3} \kappa_{\lambda}}{3} - \frac{\kappa_{V}^{2} \kappa_{\lambda}^{2}}{9}\right) + A_{1} \left(2 \kappa_{2V}^{2} - 8 \kappa_{2V} \kappa_{V}^{2} + 3 \kappa_{2V} \kappa_{V} \kappa_{\lambda} + 6 \kappa_{V}^{4} - 3 \kappa_{V}^{3} \kappa_{\lambda}\right) + A_{2} \left(- 4 \kappa_{2V}^{2} + 20 \kappa_{2V} \kappa_{V}^{2} - 8 \kappa_{2V} \kappa_{V} \kappa_{\lambda} - 16 \kappa_{V}^{4} + 8 \kappa_{V}^{3} \kappa_{\lambda}\right) + A_{3} \left(16 \kappa_{2V} \kappa_{V}^{2} - 16 \kappa_{2V} \kappa_{V} \kappa_{\lambda} - 16 \kappa_{V}^{4} + 16 \kappa_{V}^{3} \kappa_{\lambda}\right) + A_{4} \left(\frac{4 \kappa_{2V} \kappa_{V}^{2}}{5} - \frac{4 \kappa_{2V} \kappa_{V} \kappa_{\lambda}}{5} + \frac{\kappa_{V}^{4}}{5} - \frac{3 \kappa_{V}^{3} \kappa_{\lambda}}{10} + \frac{\kappa_{V}^{2} \kappa_{\lambda}^{2}}{10}\right) + A_{5} \left(- \frac{\kappa_{2V} \kappa_{V}^{2}}{45} + \frac{\kappa_{2V} \kappa_{V} \kappa_{\lambda}}{45} + \frac{\kappa_{V}^{4}}{45} - \frac{\kappa_{V}^{3} \kappa_{\lambda}}{30} + \frac{\kappa_{V}^{2} \kappa_{\lambda}^{2}}{90}\right)
$
}
        \end{column}
    \end{columns}
}


    \section{Conclusion}
    \frame{
        \frametitle{Conclusions}
        \begin{itemize} {
            \item NNT-based 3D reweighting is now built into limit-setting framework, just waiting for production to finish
            \item 2D $\kvv,\kl$ scan is already functional, and 2D plots are nearly functional
            \item 3D scanning is soon to come!
        } \end{itemize}
    }

    \announcesection{Backup}
    \fullscreenimage{}{coupling_scan_auto_chosenR0_hash_max}
    \fullscreenimage{}{coupling_scan_auto_chosenR1_hash_max}
    \fullscreenimage{}{coupling_scan_auto_chosenR2_hash_max}
    \fullscreenimage{}{coupling_scan_auto_chosen_reco_R0_hash_max}
    \fullscreenimage{Cheat Reweighting}{c2v_scan_scan_test003_samps_vbf_pd_16_xsec}

\end{document}
















