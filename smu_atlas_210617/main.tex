\documentclass{beamer}
\usepackage{pdfpages}
%Imports and customization
\usepackage{tikz}
\usepackage{graphicx}
\usepackage{tikz-feynman}
\graphicspath{ 
    {./images/}
}

\beamertemplatenavigationsymbolsempty
\setbeamertemplate{sidebar right}{}
\setbeamertemplate{footline}{
    \hfill\usebeamertemplate***{navigation symbols}
    \hspace{1cm}\insertframenumber{}/\inserttotalframenumber
}
\setbeamertemplate{caption}{\raggedright\insertcaption\par}
\setbeamersize{text margin left=5mm,text margin right=5mm} 

\setbeamerfont{itemize/enumerate body}{size=\scriptsize}
\setbeamerfont{itemize/enumerate subbody}{size=\scriptsize}
\setbeamerfont{itemize/enumerate subsubbody}{size=\scriptsize}


%Custom Macros
\newcommand{\statwarn}{
    \tiny \color{red} Absolute numbers here mean NOTHING. Plots are based on small (100k events) samples, and are highly biased. All that matters is relative position!
}


\newcommand{\fullscreenimage}[2]{
    \frame{
        \frametitle{#1} 
        \begin{figure}
        \includegraphics[height=0.9\textheight,keepaspectratio]{#2}
        \end{figure}
    }
}


\newcommand{\displayone}[3]{
    \frame{
        \frametitle{#1} 
        \begin{columns}
            \begin{column}{0.5\textwidth}
                #2
            \end{column}
            \begin{column}{0.5\textwidth}
                \begin{figure}
                    \includegraphics[width=\linewidth,height=\textheight,keepaspectratio]{#3}
                \end{figure}
            \end{column}
        \end{columns}
    }
}

\newcommand{\displayonelarge}[3]{
    \frame{
        \frametitle{#1} 
        \begin{columns}
            \begin{column}{0.3\textwidth}
                #2
            \end{column}
            \begin{column}{0.7\textwidth}
                \begin{figure}
                    \includegraphics[width=\linewidth,height=\textheight,keepaspectratio]{#3}
                \end{figure}
            \end{column}
        \end{columns}
    }
}


\newcommand{\displaytwo}[4]{
    \frame{
        \frametitle{#1} 
        #2
        \begin{columns}
            \begin{column}{0.5\textwidth}
                \begin{figure}
                    \includegraphics[width=\linewidth,height=\textheight,keepaspectratio]{#3}
                \end{figure}
            \end{column}
            \begin{column}{0.5\textwidth}
                \begin{figure}
                    \includegraphics[width=\linewidth,height=\textheight,keepaspectratio]{#4}
                \end{figure}
            \end{column}
        \end{columns}
    }
}


\newcommand{\displaythree}[5]{
    \frame{
        \begin{columns}[T]
            \begin{column}{0.5\textwidth}
                \insertframetitle{#1}\\
                #2
            \end{column}
            \begin{column}{0.5\textwidth}
                \begin{figure}
                    \includegraphics[width=\linewidth,height=\textheight,keepaspectratio]{#3}
                \end{figure}
            \end{column}
        \end{columns}
        \begin{columns}[T]
            \begin{column}{0.5\textwidth}
                \begin{figure}
                    \includegraphics[width=\linewidth,height=\textheight,keepaspectratio]{#4}
                \end{figure}
            \end{column}
            \begin{column}{0.5\textwidth}
                \begin{figure}
                    \includegraphics[width=\linewidth,height=\textheight,keepaspectratio]{#5}
                \end{figure}
            \end{column}
        \end{columns}
    }
}


\newcommand{\announcesection}[1]{
    \section{#1}
    \frame{
        \begin{center}
            {\huge #1} 
        \end{center}
    }
}


%Begin Presentation
\begin{document}
    \setbeamercolor{background canvas}{bg=}
    \title{SMU ATLAS Status Update\\VBF HH MC Combination Studies}
    \author{Chris Milke}
    \date{17 June, 2021}

    \frame{\titlepage}
    \frame{\frametitle{Overview} \tableofcontents}

    % Intro
    \section{ATLAS Di-Higgs Analysis}

\displaytwocaption{ATLAS Di-Higgs Analysis}{
    Working with Di-Higgs analysis to discover HH process.
}
{ggF_diagrams}
{$\sigma_{ggF \rightarrow HH}=33.5^{+2.4}_{-2.8}$fb at NNLO}
{vbf-hh_diagrams}
{$\sigma_{VBF \rightarrow HH}=1.73\pm0.04$fb at N\textsuperscript{3}LO}

\displayonelarge{Sensitivity}{
    Currently, no statistically significant excess over background prediction has been observed.

    \vspace{5mm}
    {\tiny Data on right from 2015-2016 period.}
}{sensitivity}


\frame{
    \frametitle{Limits} 
    \begin{columns}[T]
        \begin{column}{0.4\textwidth}
            {\scriptsize Prior and current sensitivity is too low for HH discovery, and in its place limits are being set on the HH couplings.}
        \end{column}
        \begin{column}{0.4\textwidth}{\scriptsize
            Latest observed intervals at 95\% CL: 
            \vspace{2mm}

            $-5.0 < \kl < 12.0$

            \vspace{2mm}

            $-0.76 < \kvv < 2.90$
        }\end{column}
    \end{columns}
    \begin{columns}[T]
        \begin{column}{0.4\textwidth}
            \begin{figure}
                \includegraphics[width=\linewidth,height=\textheight,keepaspectratio]{combined_kl_limits}
                \caption{\tiny Public combined limits on $\kl$, June 2019;
                    data from 2015-2016 period.}
            \end{figure}
        \end{column}
        \begin{column}{0.4\textwidth}
            \begin{figure}
                \includegraphics[width=\linewidth,height=\textheight,keepaspectratio]{c2v_limits_paper}
                \caption{\tiny VBF->4b public limits on $\kvv$, 
                    January 2020;
                    data from 2016-2018 period.}
            \end{figure}
        \end{column}
    \end{columns}
}


\displayone{VBF -> HH -> 4b}{
    My research is on the 4b decay mode specifically, though some work has been done for other VBF groups as well.

    \vspace{10mm}

    \begin{center}
    \resizebox{0.30\textheight}{!}{\begin{tikzpicture} \begin{feynman}
    \vertex (c2v) {\tiny $C_{2V}$};
    \vertex [above right=of c2v] (h1) {H};
    \vertex [below right=of c2v] (h2) {H};
    \vertex [above left=of c2v] (vb1);
    \vertex [below left=of c2v] (vb2);
    \vertex [left=of vb1] (q1i) {q};
    \vertex [left=of vb2] (q2i) {q};
    \vertex [above right=of h1] (b1) {$b$};
    \vertex [below right=of h2] (bbar2) {$\bar b$};
    \vertex [below=of b1] (bbar1) {$\bar b$};
    \vertex [above=of bbar2] (b2) {$b$};

    \vertex [above=of b1] (q1f) {q};
    \vertex [below=of bbar2] (q2f) {q};

    \diagram* {
        (q1i) -- (vb1) -- (q1f),
        (q2i) -- (vb2) -- (q2f), 
        (vb1) -- [boson] (c2v) -- [boson] (vb2),
        (h1) -- [scalar] (c2v) -- [scalar] (h2),
        (b1) -- (h1) -- (bbar1),
        (b2) -- (h2) -- (bbar2),
    };
\end{feynman} \end{tikzpicture}
}
    \end{center}
}
{branching_ratios}

    \input{combination.tex}
    \section{Applying Linear Combination}

\frame{
    \frametitle{Linear Combination Using Post Reconstruction/Selection Samples}
    \begin{columns}
        \begin{column}{0.4\textwidth}
            \begin{center} 
            {\tiny Signal Sample Basis Set}

            \resizebox{0.2\textheight}{!}{ \begin{tabular}{ |l|l|l| }
                \hline
                \textbf {$\kappa_{2V}$} & \textbf {$\kappa_\lambda$} & \textbf {$\kappa_V$} \\
                \hline
                    1.  &   1. & 1.  \\
                    2.  &   1. & 1.  \\
                    1.5 &   1. & 1.  \\
                    0.  &   1. & 0.5 \\
                    1.  &   0. & 1.  \\
                    1.  &  10. & 1.  \\
                \hline
            \end{tabular}}
            \end{center}

        \end{column}
        \begin{column}{0.6\textwidth}
            \resizebox{0.8\textwidth}{!}{ \begin{minipage}{1.0\textwidth}
            Linear Combination Equation

            \vspace{10mm}

            {\tiny \input{final_amplitude_current_3D_reco.tex}}
            \end{minipage}}
        \end{column}
    \end{columns}
}

\displaythree{Validity of Sample Combinations}
{ \small 
    Combining the six basis samples allows for modelling of distributions for any coupling values.
}
{reco_mHH_cvv1p0cl2p0cv1p0_ancient}
{reco_mHH_cvv0p0cl1p0cv1p0_ancient}
{reco_mHH_cvv4p0cl1p0cv1p0_ancient}

    \section{Negative Weight Metric}

\displaytwo{Negative Weights}{
    Negative weights can appear in the $m_{HH}$ combination. These are unphysical and should be avoided.

    In further regions of the $\kappa$ coupling space, these negative weights can be become dangerously common.
}{reco_mHH_cvv1p0cl2p0cv1p0_ancient}
{preview_reco_mHH_cvv0p0cl-9p0cv1p0_old}

\displayonelarge{Assessing Basis Performance via Negative Weight Integral}{
    Take the surface integral of the number of negative bins at each point at in the $\kappa$ parameter space (mutliplying by the $\kvv \times \kl$ ``area") as a general metric of performance.
}{negative_weights_rank027}

\displaythree{Negative-Weight Map of Different Bases}{
    Other bases show considerable improvement in the number of negative bins
}
{negative_weights_rank027}
{negative_weights_rank015}
{negative_weights_rank000}


\displaytwocaption{Limits with Negative-Weight Integral Rank-1 Basis}{
    Fewer negative weights results in a much more ``natural" limit boundary.

    Sharp edges and corners appear to have been a consequence of poor signal modelling.
}
{2D_scan_2D_scan_test01_noMC16e_samps_vbf_pd_1617_c1v1.0_exclusion}
{Using old basis}
{2D_scan_2D_scan_test02_newBasis_samps_vbf_pd_1617_c1v1.0_exclusion}
{Using new basis}

    \section{New MC Samples}

\frame{
    \frametitle{New MC Samples!}
    \begin{center} 
        {\small
            12 New MC Samples,\\
            619 Possible Combinations of 6,\\
            250 Include SM Sample
        }
        \vspace{5mm}

        \resizebox{0.8\textheight}{!}{
        \begin{columns} \begin{column}{0.3\textwidth}
            \begin{tabular}{ |l|l|l| }
                \hline
                \textbf {$\kappa_{2V}$} & \textbf {$\kappa_\lambda$} & \textbf {$\kappa_V$} \\
                \hline
                0    & 0   & 1   \\
                0    & 1   & 1   \\
                0.5  & 1   & 1   \\
                1    & 0   & 1   \\
                \hline
            \end{tabular}
        \end{column} \begin{column}{0.3\textwidth}
            \begin{tabular}{ |l|l|l| }
                \hline
                \textbf {$\kappa_{2V}$} & \textbf {$\kappa_\lambda$} & \textbf {$\kappa_V$} \\
                \hline
                1    & 1   & 0.5 \\
                1    & 1   & 1   \\
                1    & 1   & 1.5 \\
                1    & 10  & 1   \\
                \hline
            \end{tabular}
        \end{column} \begin{column}{0.3\textwidth}
            \begin{tabular}{ |l|l|l| }
                \hline
                \textbf {$\kappa_{2V}$} & \textbf {$\kappa_\lambda$} & \textbf {$\kappa_V$} \\
                \hline
                1    & 2   & 1   \\
                1.5  & 1   & 1   \\
                2    & 1   & 1   \\
                3    & 1   & 1   \\
                \hline
            \end{tabular}
        \end{column} \end{columns}
        }
    \end{center}
    \vspace{7mm}

    \begin{center} 
    Top 10 New Bases

    \resizebox{0.6\textwidth}{!}{
    \begin{tabular}{ |l|l|l|l|l|l|c| }
        \hline
            \textbf {Sample 1} & \textbf {Sample 2} & \textbf {Sample 3} &
            \textbf {Sample 4} & \textbf {Sample 5} & \textbf {Sample 6} &
            \textbf {Nweight Int}\\
            \hline
            (1, 1, 1) & (0.5, 1, 1) & (3, 1, 1) & (1, 2, 1 ) & (1, 10, 1) & (0, 0, 1) &  323 \\
            (1, 1, 1) & (0.5, 1, 1) & (3, 1, 1) & (1, 0, 1 ) & (1, 10, 1) & (0, 0, 1) &  343 \\
            (1, 1, 1) & (0.5, 1, 1) & (1.5, 1, 1) & (1, 2, 1 ) & (1, 10, 1) & (0, 0, 1) &  362 \\
            (1, 1, 1) & (0.5, 1, 1) & (2, 1, 1) & (1, 2, 1 ) & (1, 10, 1) & (0, 0, 1) &  365 \\
            (1, 1, 1) & (0.5, 1, 1) & (1.5, 1, 1) & (1, 0, 1 ) & (1, 10, 1) & (0, 0, 1) &  387 \\
            (1, 1, 1) & (0, 1, 1) & (1, 2, 1) & (1, 10, 1) & (1, 1, 0.5)  & (0, 0, 1) &  390 \\
            (1, 1, 1) & (0.5, 1, 1) & (2, 1, 1) & (1, 0, 1 ) & (1, 10, 1) & (0, 0, 1) &  394 \\
            (1, 1, 1) & (0.5, 1, 1) & (1, 2, 1) & (1, 10, 1) & (1, 1, 0.5)  & (0, 0, 1) &  397 \\
            (1, 1, 1) & (0.5, 1, 1) & (1, 0, 1) & (1, 10, 1) & (1, 1, 0.5)  & (0, 0, 1) &  398 \\
            (1, 1, 1) & (0, 1, 1) & (3, 1, 1) & (1, 2, 1 ) & (1, 10, 1) & (0, 0, 1) &  400 \\
            \hline
    \end{tabular} }
    \end{center}
}

\frame{ \frametitle{6-Term Top Two Bases}
    Rank 1\\
    \vspace{3mm}
    \resizebox{0.8\textwidth}{!}{ \begin{minipage}{1.0\textwidth}
    {\tiny $\left(2 \kappa_{2V}^{2} - 2 \kappa_{2V} \kappa_{V}^{2} - 3 \kappa_{2V} \kappa_{V} \kappa_{\lambda} + 3 \kappa_{V}^{3} \kappa_{\lambda}\right) \times \sigma{\left(\frac{1}{2},1,1 \right)} +$

$ \left(2 \kappa_{2V}^{2} - 2 \kappa_{2V} \kappa_{V}^{2} - \kappa_{2V} \kappa_{V} \kappa_{\lambda} + \kappa_{V}^{3} \kappa_{\lambda}\right) \times \sigma{\left(\frac{3}{2},1,1 \right)} +$

$ \left(- \frac{5 \kappa_{2V} \kappa_{V}^{2}}{4} + \frac{5 \kappa_{2V} \kappa_{V} \kappa_{\lambda}}{4} + \frac{\kappa_{V}^{3} \kappa_{\lambda}}{8} - \frac{\kappa_{V}^{2} \kappa_{\lambda}^{2}}{8}\right) \times \sigma{\left(1,2,1 \right)} +$

$ \left(- \kappa_{2V} \kappa_{V}^{2} + \kappa_{2V} \kappa_{V} \kappa_{\lambda} + \kappa_{V}^{4} - \kappa_{V}^{3} \kappa_{\lambda}\right) \times \sigma{\left(0,0,1 \right)} +$

$ \left(\frac{\kappa_{2V} \kappa_{V}^{2}}{36} - \frac{\kappa_{2V} \kappa_{V} \kappa_{\lambda}}{36} - \frac{\kappa_{V}^{3} \kappa_{\lambda}}{72} + \frac{\kappa_{V}^{2} \kappa_{\lambda}^{2}}{72}\right) \times \sigma{\left(1,10,1 \right)} +$

$ \left(- 4 \kappa_{2V}^{2} + \frac{56 \kappa_{2V} \kappa_{V}^{2}}{9} + \frac{16 \kappa_{2V} \kappa_{V} \kappa_{\lambda}}{9} - \frac{28 \kappa_{V}^{3} \kappa_{\lambda}}{9} + \frac{\kappa_{V}^{2} \kappa_{\lambda}^{2}}{9}\right) \times \sigma{\left(1,1,1 \right)}$
}
    \end{minipage}}

    \vspace{7mm}

    Rank 2\\
    \vspace{3mm}
    \resizebox{0.8\textwidth}{!}{ \begin{minipage}{1.0\textwidth}
    {\tiny \input{final_amplitude_optimal_3DR1.tex}}
    \end{minipage}}
}

\displaythree{6-Term Negative Weight Heatmap}
{ \tiny
    New production (below) performs signficantly \textit{worse} than old production (right).
    \vspace{3mm}

    I think this is to do with the absence of the $\kvv,\kl,\kv = [0, 1, 0.5]$ sample, but this is unclear.
}
{old_negative_weights_toprank000}
{negative_weights_toprank0}
{negative_weights_toprank1}


    \section{Conclusion}
    \displaytwocaption{Conclusion}{ \tiny
        Proposed solution is to produce one new MC sample which can hopefully stabilize the combinations.
        I have provided the following two options as recommendations,
            with the below plots made at \textit{truth-level} to show usefulness of the samples. 

        Both of these samples look like they would drastically improve performance of coupling scans.
    }
    {negative_weights_iterA00_truth_quadrank0}
    { Using Sample $(\kvv=1, \kl=-5, \kv=0.5)$ }
    {negative_weights_iterA02_truth_quadrank1}
    { Using Sample $(\kvv=3, \kl=-9, \kv=1)$ }
\end{document}
