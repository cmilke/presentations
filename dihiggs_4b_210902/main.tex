\documentclass{beamer}
\usepackage{pdfpages}
%Imports and customization
\usepackage{tikz}
\usepackage{graphicx}
\usepackage{tikz-feynman}
\graphicspath{ 
    {./images/}
}

\beamertemplatenavigationsymbolsempty
\setbeamertemplate{sidebar right}{}
\setbeamertemplate{footline}{
    \hfill\usebeamertemplate***{navigation symbols}
    \hspace{1cm}\insertframenumber{}/\inserttotalframenumber
}
\setbeamertemplate{caption}{\raggedright\insertcaption\par}
\setbeamersize{text margin left=5mm,text margin right=5mm} 

\setbeamerfont{itemize/enumerate body}{size=\scriptsize}
\setbeamerfont{itemize/enumerate subbody}{size=\scriptsize}
\setbeamerfont{itemize/enumerate subsubbody}{size=\scriptsize}


%Custom Macros
\newcommand{\statwarn}{
    \tiny \color{red} Absolute numbers here mean NOTHING. Plots are based on small (100k events) samples, and are highly biased. All that matters is relative position!
}


\newcommand{\fullscreenimage}[2]{
    \frame{
        \frametitle{#1} 
        \begin{figure}
        \includegraphics[height=0.9\textheight,keepaspectratio]{#2}
        \end{figure}
    }
}


\newcommand{\displayone}[3]{
    \frame{
        \frametitle{#1} 
        \begin{columns}
            \begin{column}{0.5\textwidth}
                #2
            \end{column}
            \begin{column}{0.5\textwidth}
                \begin{figure}
                    \includegraphics[width=\linewidth,height=\textheight,keepaspectratio]{#3}
                \end{figure}
            \end{column}
        \end{columns}
    }
}

\newcommand{\displayonelarge}[3]{
    \frame{
        \frametitle{#1} 
        \begin{columns}
            \begin{column}{0.3\textwidth}
                #2
            \end{column}
            \begin{column}{0.7\textwidth}
                \begin{figure}
                    \includegraphics[width=\linewidth,height=\textheight,keepaspectratio]{#3}
                \end{figure}
            \end{column}
        \end{columns}
    }
}


\newcommand{\displaytwo}[4]{
    \frame{
        \frametitle{#1} 
        #2
        \begin{columns}
            \begin{column}{0.5\textwidth}
                \begin{figure}
                    \includegraphics[width=\linewidth,height=\textheight,keepaspectratio]{#3}
                \end{figure}
            \end{column}
            \begin{column}{0.5\textwidth}
                \begin{figure}
                    \includegraphics[width=\linewidth,height=\textheight,keepaspectratio]{#4}
                \end{figure}
            \end{column}
        \end{columns}
    }
}


\newcommand{\displaythree}[5]{
    \frame{
        \begin{columns}[T]
            \begin{column}{0.5\textwidth}
                \insertframetitle{#1}\\
                #2
            \end{column}
            \begin{column}{0.5\textwidth}
                \begin{figure}
                    \includegraphics[width=\linewidth,height=\textheight,keepaspectratio]{#3}
                \end{figure}
            \end{column}
        \end{columns}
        \begin{columns}[T]
            \begin{column}{0.5\textwidth}
                \begin{figure}
                    \includegraphics[width=\linewidth,height=\textheight,keepaspectratio]{#4}
                \end{figure}
            \end{column}
            \begin{column}{0.5\textwidth}
                \begin{figure}
                    \includegraphics[width=\linewidth,height=\textheight,keepaspectratio]{#5}
                \end{figure}
            \end{column}
        \end{columns}
    }
}


\newcommand{\announcesection}[1]{
    \section{#1}
    \frame{
        \begin{center}
            {\huge #1} 
        \end{center}
    }
}


%Begin Presentation
\begin{document}
    \setbeamercolor{background canvas}{bg=}
    \title{VBF 3D Limits Update}
    \author{Chris Milke}
    \date{2 September, 2021}

    \frame{\titlepage}
    \frame{\frametitle{Overview} \tableofcontents}

    % Show the old math intro again
    \section{New MC Signal Sample}
\displaythree{Last Time...}
{ \small
    New 2021 production (below) performed signficantly \textit{worse} than old 2020 production (right).
}
{old_negative_weights_toprank000}
{negative_weights_toprank0}
{negative_weights_toprank1}

\displayfour{\small Poor Basis Options Manifested as Very Poor Signal Modelling}
{reco_mHH_compare_preview_auto_top_3D_0-1_cvv-1p5cl14p0cv1p0}
{reco_mHH_compare_preview_auto_top_3D_0-1_cvv0p5cl5p0cv1p0}
{reco_mHH_compare_preview_auto_top_3D_0-1_cvv2p0cl-3p0cv1p0}
{reco_mHH_compare_preview_auto_top_3D_0-1_cvv2p0cl-9p0cv1p0}

    \input{new_sample.tex}
    \announcesection{Validation and Signal Checking}

\displaythree{Validation of May VS August Set - Pt1}
{In the near-SM regime, the new August Basis shows slightly more deviation
    from the generated sample than the May Basis}
{/validation_dump/reco_mHH_compare_validate_2021newold_cvv0p0cl0p0cv1p0}
{/validation_dump/reco_mHH_compare_validate_2021newold_cvv0p0cl1p0cv1p0}
{/validation_dump/reco_mHH_compare_validate_2021newold_cvv0p5cl1p0cv1p0}

\displaythree{Validation of May VS August Set - Pt2}{}
{/validation_dump/reco_mHH_compare_validate_2021newold_cvv1p0cl0p0cv1p0}
{/validation_dump/reco_mHH_compare_validate_2021newold_cvv1p0cl10p0cv1p0}
{/validation_dump/reco_mHH_compare_validate_2021newold_cvv1p0cl1p0cv0p5}

\displaythree{Validation of May VS August Set - Pt3}{}
{/validation_dump/reco_mHH_compare_validate_2021newold_cvv1p0cl1p0cv1p0}
{/validation_dump/reco_mHH_compare_validate_2021newold_cvv1p0cl1p0cv1p5}
{/validation_dump/reco_mHH_compare_validate_2021newold_cvv1p0cl2p0cv1p0}

\displayfour{Validation of May VS August Set - Pt4}
{/validation_dump/reco_mHH_compare_validate_2021newold_cvv1p0cl-5p0cv0p5}
{/validation_dump/reco_mHH_compare_validate_2021newold_cvv1p5cl1p0cv1p0}
{/validation_dump/reco_mHH_compare_validate_2021newold_cvv2p0cl1p0cv1p0}
{/validation_dump/reco_mHH_compare_validate_2021newold_cvv3p0cl1p0cv1p0}



\announcesection{BSM Previews}

\displaytwo{May VS August Basis Distributions $\kv=1$ -  Pt1}{

}
{preview_dump/reco_mHH_compare_preview_2021newold_cvv0p50cl14p00cv1p00}
{preview_dump/reco_mHH_compare_preview_2021newold_cvv0p50cl1p00cv1p00}

\displayfour{May VS August Basis Distributions $\kv=1$ -  Pt2}
{preview_dump/reco_mHH_compare_preview_2021newold_cvv0p50cl-3p00cv1p00}
{preview_dump/reco_mHH_compare_preview_2021newold_cvv0p50cl5p00cv1p00}
{preview_dump/reco_mHH_compare_preview_2021newold_cvv0p50cl-7p00cv1p00}
{preview_dump/reco_mHH_compare_preview_2021newold_cvv0p50cl-9p00cv1p00}

\displayfour{May VS August Basis Distributions $\kv=1$ -  Pt3}
{preview_dump/reco_mHH_compare_preview_2021newold_cvv1p00cl14p00cv1p00}
{preview_dump/reco_mHH_compare_preview_2021newold_cvv1p00cl1p00cv1p00}
{preview_dump/reco_mHH_compare_preview_2021newold_cvv1p00cl-3p00cv1p00}
{preview_dump/reco_mHH_compare_preview_2021newold_cvv1p00cl5p00cv1p00}

\displayfour{May VS August Basis Distributions $\kv=1$ -  Pt4}
{preview_dump/reco_mHH_compare_preview_2021newold_cvv1p00cl-7p00cv1p00}
{preview_dump/reco_mHH_compare_preview_2021newold_cvv1p00cl-9p00cv1p00}
{preview_dump/reco_mHH_compare_preview_2021newold_cvv-1p50cl14p00cv1p00}
{preview_dump/reco_mHH_compare_preview_2021newold_cvv-1p50cl1p00cv1p00}

\displayfour{May VS August Basis Distributions $\kv=1$ -  Pt5}
{preview_dump/reco_mHH_compare_preview_2021newold_cvv-1p50cl-3p00cv1p00}
{preview_dump/reco_mHH_compare_preview_2021newold_cvv-1p50cl5p00cv1p00}
{preview_dump/reco_mHH_compare_preview_2021newold_cvv-1p50cl-7p00cv1p00}
{preview_dump/reco_mHH_compare_preview_2021newold_cvv-1p50cl-9p00cv1p00}

\displayfour{May VS August Basis Distributions $\kv=1$ -  Pt6}
{preview_dump/reco_mHH_compare_preview_2021newold_cvv2p00cl14p00cv1p00}
{preview_dump/reco_mHH_compare_preview_2021newold_cvv2p00cl1p00cv1p00}
{preview_dump/reco_mHH_compare_preview_2021newold_cvv2p00cl-3p00cv1p00}
{preview_dump/reco_mHH_compare_preview_2021newold_cvv2p00cl5p00cv1p00}

\displayfour{May VS August Basis Distributions $\kv=1$ -  Pt7}
{preview_dump/reco_mHH_compare_preview_2021newold_cvv2p00cl-7p00cv1p00}
{preview_dump/reco_mHH_compare_preview_2021newold_cvv2p00cl-9p00cv1p00}
{preview_dump/reco_mHH_compare_preview_2021newold_cvv3p50cl14p00cv1p00}
{preview_dump/reco_mHH_compare_preview_2021newold_cvv3p50cl1p00cv1p00}

\displayfour{May VS August Basis Distributions $\kv=1$ -  Pt8}
{preview_dump/reco_mHH_compare_preview_2021newold_cvv3p50cl-3p00cv1p00}
{preview_dump/reco_mHH_compare_preview_2021newold_cvv3p50cl5p00cv1p00}
{preview_dump/reco_mHH_compare_preview_2021newold_cvv3p50cl-7p00cv1p00}
{preview_dump/reco_mHH_compare_preview_2021newold_cvv3p50cl-9p00cv1p00}


    \frame{
    \begin{columns}[T]
        \begin{column}{0.4\textwidth}
            {\usebeamercolor[fg]{title} \insertframetitle{2019 2D Limits} }\\
            \vspace{5mm}
            Limits....
        \end{column}
        \begin{column}{0.4\textwidth}
            \begin{figure}
                \includegraphics[width=\linewidth,height=0.35\textheight,keepaspectratio]{
                    2D_scan_2D_scan_test02_newBasis_samps_vbf_pd_1617_c1v1.0_exclusion}
                \caption{\tiny 2020 Basis, 2015,2016,2017 only}
            \end{figure}
        \end{column}
    \end{columns}
    \vspace{0.5\textheight}
}


\frame{
    \begin{columns}[T]
        \begin{column}{0.4\textwidth}
            {\usebeamercolor[fg]{title} \insertframetitle{2019 2D Limits} }\\
            \vspace{5mm}
            Limits....
        \end{column}
        \begin{column}{0.4\textwidth}
            \begin{figure}
                \includegraphics[width=\linewidth,height=0.35\textheight,keepaspectratio]
                    {2D_scan_2D_scan_test02_newBasis_samps_vbf_pd_1617_c1v1.0_exclusion}
                \caption{\tiny 2020 Basis, 2015,2016,2017 only}
            \end{figure}
        \end{column}
    \end{columns}
    \begin{columns}[T]
        \begin{column}{0.4\textwidth}
            \begin{figure}
                \includegraphics[width=\linewidth,height=0.35\textheight,keepaspectratio]
                    {2D_scan_full_3D_scan_2021_samps_vbf_pd_161718_k1v1_exclusion}
                \caption{2021 May Basis, Full Data}
            \end{figure}
        \end{column}
        \begin{column}{0.4\textwidth}
        \end{column}
    \end{columns}
}


\frame{
    \begin{columns}[T]
        \begin{column}{0.4\textwidth}
            {\usebeamercolor[fg]{title} \insertframetitle{2019 2D Limits} }\\
            \vspace{5mm}
            Limits....
        \end{column}
        \begin{column}{0.4\textwidth}
            \begin{figure}
                \includegraphics[width=\linewidth,height=0.35\textheight,keepaspectratio]
                    {2D_scan_2D_scan_test02_newBasis_samps_vbf_pd_1617_c1v1.0_exclusion}
                \caption{\tiny 2020 Basis, 2015,2016,2017 only}
            \end{figure}
        \end{column}
    \end{columns}
    \begin{columns}[T]
        \begin{column}{0.4\textwidth}
            \begin{figure}
                \includegraphics[width=\linewidth,height=0.35\textheight,keepaspectratio]
                    {2D_scan_full_3D_scan_2021_samps_vbf_pd_161718_k1v1_exclusion}
                \caption{\tiny 2021 May Basis, Full Data}
            \end{figure}
        \end{column}
        \begin{column}{0.4\textwidth}
            \begin{figure}
                \includegraphics[width=\linewidth,height=0.35\textheight,keepaspectratio]
                    {new_limits/2D_scan_full_3D_scan_NEO2021_samps_vbf_pd_161718_k1v1_exclusion_coarse}
                \caption{\tiny 2021 May Basis, Full Data}
            \end{figure}
        \end{column}
    \end{columns}
}
\displayfour{Previously Excluded Regions Closer Look}
{reco_mHH_compare_preview_2021newold_cvv0p00cl10p00cv1p00}
{reco_mHH_compare_preview_2021newold_cvv0p00cl13p00cv1p00}
{reco_mHH_compare_preview_2021newold_cvv2p00cl-10p00cv1p00}
{reco_mHH_compare_preview_2021newold_cvv2p50cl-10p00cv1p00}

\fullscreenimage{2021 May Basis, Higher Fitting Resolution}
{new_limits/2D_scan_full_3D_scan_NEO2021_samps_vbf_pd_161718_k1v1_exclusion}



    %Conclusion 
    \section{Conclusion}
    \displayonelarge{Conclusion}{
        \begin{itemize}
            \item New MC sample point allows for far more stable sample combinations
            \item New Basis produces signal distributions that appear very reasonable
            \item Limits with the new basis are looser in $\kl$,
                which is unfortunate, but probably more accurate
        \end{itemize}
    }{new_limits/2D_scan_full_3D_scan_NEO2021_samps_vbf_pd_161718_k1v_multislice_exclusion}


    \announcesection{Backup}
    \fullscreenimage{$\kvv$ Basic Scan Plot}{k2v_scan_full_3D_scan_2021_samps_vbf_pd_161718_kl_1.0_xsec}
    \section{Linear Algebra}

% show diagrams with terms, then show squaring
\frame{
    \frametitle{3D Coupling Dependence}
    \begin{figure}
    \includegraphics[width=0.6\linewidth,height=0.6\textheight,keepaspectratio]{vbf-hh_diagrams2b}
    \end{figure}
    $ \sigma = | \kv \kl M_s + \kv^2 M_t + \kvv M_x |^2 = $

    \vspace{10mm}

    $ \kv^2 \kl^2 M_s^2 + \kv^4 M_t^2 + \kvv^2 M_x^2 
    + \kv^3 \kl (M_s^* M_t + M_t^* M_s) 
    + \kv \kl \kvv (M_s^* M_x + M_x^* M_s ) 
    + \kv^2 \kvv (M_t^* M_x + M_x^* M_t )$

    \vspace{10mm}

    $ \sigma = \kv^2 \kl^2 a_1 + \kv^4 a_2 + \kvv^2 a_3 + \kv^3 \kl a_4 + \kv \kl \kvv a_5 + \kv^2 \kvv a_6 $
}

    \fullscreenimage{$\kv$ Slice nWeight Map}{multislice_negative_weightsneo}
    

\announcesection{More BSM (kv=1.04) Previews}

\displaytwo{May VS August Basis Distributions $\kv=1.04$ -  Pt1}{}
{preview_dump/reco_mHH_compare_preview_2021newold_cvv0p50cl14p00cv1p04}
{preview_dump/reco_mHH_compare_preview_2021newold_cvv0p50cl1p00cv1p04}

\displayfour{May VS August Basis Distributions $\kv=1.04$ -  Pt2}
{preview_dump/reco_mHH_compare_preview_2021newold_cvv0p50cl-3p00cv1p04}
{preview_dump/reco_mHH_compare_preview_2021newold_cvv0p50cl5p00cv1p04}
{preview_dump/reco_mHH_compare_preview_2021newold_cvv0p50cl-7p00cv1p04}
{preview_dump/reco_mHH_compare_preview_2021newold_cvv0p50cl-9p00cv1p04}

\displayfour{May VS August Basis Distributions $\kv=1.04$ -  Pt3}
{preview_dump/reco_mHH_compare_preview_2021newold_cvv1p00cl14p00cv1p04}
{preview_dump/reco_mHH_compare_preview_2021newold_cvv1p00cl1p00cv1p04}
{preview_dump/reco_mHH_compare_preview_2021newold_cvv1p00cl-3p00cv1p04}
{preview_dump/reco_mHH_compare_preview_2021newold_cvv1p00cl5p00cv1p04}

\displayfour{May VS August Basis Distributions $\kv=1.04$ -  Pt4}
{preview_dump/reco_mHH_compare_preview_2021newold_cvv1p00cl-7p00cv1p04}
{preview_dump/reco_mHH_compare_preview_2021newold_cvv1p00cl-9p00cv1p04}
{preview_dump/reco_mHH_compare_preview_2021newold_cvv-1p50cl14p00cv1p04}
{preview_dump/reco_mHH_compare_preview_2021newold_cvv-1p50cl1p00cv1p04}

\displayfour{May VS August Basis Distributions $\kv=1.04$ -  Pt5}
{preview_dump/reco_mHH_compare_preview_2021newold_cvv-1p50cl-3p00cv1p04}
{preview_dump/reco_mHH_compare_preview_2021newold_cvv-1p50cl5p00cv1p04}
{preview_dump/reco_mHH_compare_preview_2021newold_cvv-1p50cl-7p00cv1p04}
{preview_dump/reco_mHH_compare_preview_2021newold_cvv-1p50cl-9p00cv1p04}

\displayfour{May VS August Basis Distributions $\kv=1.04$ -  Pt6}
{preview_dump/reco_mHH_compare_preview_2021newold_cvv2p00cl14p00cv1p04}
{preview_dump/reco_mHH_compare_preview_2021newold_cvv2p00cl1p00cv1p04}
{preview_dump/reco_mHH_compare_preview_2021newold_cvv2p00cl-3p00cv1p04}
{preview_dump/reco_mHH_compare_preview_2021newold_cvv2p00cl5p00cv1p04}

\displayfour{May VS August Basis Distributions $\kv=1.04$ -  Pt7}
{preview_dump/reco_mHH_compare_preview_2021newold_cvv2p00cl-7p00cv1p04}
{preview_dump/reco_mHH_compare_preview_2021newold_cvv2p00cl-9p00cv1p04}
{preview_dump/reco_mHH_compare_preview_2021newold_cvv3p50cl14p00cv1p04}
{preview_dump/reco_mHH_compare_preview_2021newold_cvv3p50cl1p00cv1p04}

\displayfour{May VS August Basis Distributions $\kv=1.04$ -  Pt8}
{preview_dump/reco_mHH_compare_preview_2021newold_cvv3p50cl-3p00cv1p04}
{preview_dump/reco_mHH_compare_preview_2021newold_cvv3p50cl5p00cv1p04}
{preview_dump/reco_mHH_compare_preview_2021newold_cvv3p50cl-7p00cv1p04}
{preview_dump/reco_mHH_compare_preview_2021newold_cvv3p50cl-9p00cv1p04}


    \announcesection{Let's Try Something Else}

\frame{
    \frametitle{Closer Look at How Combination is Done}
    {\footnotesize
        The long polynomial functions are just coefficients $c_i(\kvv,\kl,\kv)$ to the x-secs $|A_i|^2 = \sigma_i$
        The linearly combined signal distribution $\tilde{\sigma}(\kvv,\kl,\kv)$ can be viewed more simply as:

    }
    \vspace{5mm}

    $ \tilde{\sigma}(\kvv,\kl,\kv) = $
    \vspace{3mm}
    \begin{columns}
        \begin{column}{0.55\textwidth}
            \resizebox{0.8\textwidth}{!}{ \begin{minipage}{1.0\textwidth}
            {\tiny \input{final_amplitude_current_3D_reco.tex}}
            \end{minipage}}
        \end{column}
        \begin{column}{0.05\textwidth}
            \rightarrow
        \end{column}
        \begin{column}{0.4\textwidth}
            { \small
                $c_1(\kvv,\kl,\kv) \times \sigma(1,1,1   ) +$\\
                $c_2(\kvv,\kl,\kv) \times \sigma(2,1,1   ) +$\\
                $c_3(\kvv,\kl,\kv) \times \sigma(1.5,1,1 ) +$\\
                $c_4(\kvv,\kl,\kv) \times \sigma(0,1,0.5 ) +$\\
                $c_5(\kvv,\kl,\kv) \times \sigma(1,0,1   ) +$\\
                $c_6(\kvv,\kl,\kv) \times \sigma(1,10,1  )  $\\
            \par }
        \end{column}
    \end{columns}
}


\frame{
    \frametitle{Closer Look at How Combination is Done}

    $ \tilde{\sigma}(\kvv,\kl,\kv) = $

    { \small
        \hspace{20pt} $c_1(\kvv,\kl,\kv) \times \sigma(1,1,1   ) +$\\
        \hspace{20pt} $c_2(\kvv,\kl,\kv) \times \sigma(2,1,1   ) +$\\
        \hspace{20pt} $c_3(\kvv,\kl,\kv) \times \sigma(1.5,1,1 ) +$\\
        \hspace{20pt} $c_4(\kvv,\kl,\kv) \times \sigma(0,1,0.5 ) +$\\
        \hspace{20pt} $c_5(\kvv,\kl,\kv) \times \sigma(1,0,1   ) +$\\
        \hspace{20pt} $c_6(\kvv,\kl,\kv) \times \sigma(1,10,1  )  $\\
    \par }
    \vspace{5mm}

    { \footnotesize
        Neither the coefficients nor the xsecs can be looked at in isolation; they must be viewed \textit{together} as a combined product.
        \vspace{3mm}

        If the \textit{magnitude} of some $c_i\sigma_i$ products are disproportionately large for some $\kappa$ value,
        then the smaller $c_i\sigma_i$ (and their associated sample) barely contribute to the combined signal.
        \vspace{3mm}

        The combination depends on \textit{all} six samples.
        All six samples should be contributing at all points (no slackers!).
    \par }
}


\displaytwo{The Metric of Solidarity}{
    Define measure of ``closeness" of $c_i\sigma_i$ terms:
    \vspace{3mm}

    \textit{Solidarity}, $S \equiv \frac{ \sum\limits_{i=1}^6 c_i\sigma_i }{ \textrm{Stdev}(|c_i\sigma_i|) } $
    \vspace{3mm}

     {\tiny
        i.e. take the standard deviation of the \textit{absolute values} of the $c_i\sigma_i$ products,

        normalize this by their sum (the modeled x-sec at that point),

        and then take the \textit{reciprocal} of this normalized standard deviation ($S$ increases as standard deviation gets smaller).
        \par
    }


}{contribution_maxrank01}{negative_weights_toprank000}

\displayonelarge{Correlation Between Negative Weights and Solidarity}{
    Take the surface integral of the Solidarity map for every basis to produce a ``solidarity integral", and compare to Nweight Integral.
    \vspace{5mm}

    Higher Solidarity values \textit{strongly} correlate to fewer instances of negative weights
}{Nweight_integral_VS_reco_solidarity_integral}
\displayonelarge{Correlation Between Negative Weights and Theoretical Solidarity}{
    While looser, the correlation holds even when using only the VBF->HH->4b theoretical cross-section values
    \small{(NO MONTE-CARLO)}.
    \vspace{5mm}

    Can possibly be used to predict future MC productions. Further study needed.
}{Nweight_integral_VS_theory_solidarity_integral}

    \frame{
    \frametitle{Variation Scan Range}

    Wide range of values to check, brute-force approach:
    \vspace{3mm}

    \begin{itemize} {
        \item $\kvv =$ numpy.arange(-1, 4.5, 0.5)
        \item $\kl =$ numpy.arange(-9, 11, 1)
        \item $\kv =$ [ 0.5, 1, 1.5 ]
        \item 660 different variations
        \item Require all combinations to include SM and $\kl=2$ point to reduce combinatorics
        \item Each variation then has $\approx$ 227 valid possible combinations
        \item Total of $\approx$ 148,820 different combinations
    } \end{itemize}
    \vspace{3mm}

    Add each new variation to available samples one at a time,
    recalculate the solidarity integral for all possible valid combinations with the new sample in place
}

\displaytwocaption{Solidarity Histograms}{
    Each variation tested produces $\approx$ 227 combinations,
    and each combination has its own solidarity integral.
}
{projective_solidarity_all}{Overview of all 660 variations (sorted by Sum-Total of all integrals)}
{projective_solidarity_dump}{Closer look at top 10 variations}

\displaytwo{Performance Predictions}{
    Sum together the solidarity integrals of all possible combinations to produce a heatmap of potential improvement.
    \vspace{3mm}

    Improvement largely seems to come from placing a sample in the middle of the poor-performing regions
}
{negative_weights_toprank0}
{solidarity_performance_kv1}

\displaythree{Performance Predictions with Different $\kv$}{
    Variations with non-SM $\kv$ values can also potentially help
}
{solidarity_performance_kv0.5}
{solidarity_performance_kv1}
{solidarity_performance_kv1.5}



%\displayfour{Post-Selection Performance (Top) VS Truth-Level (Bottom)}
%{negative_weights_toprank0}
%{negative_weights_toprank1}
%{negative_weightsuncappedtruthRtop0}
%{negative_weightsuncappedtruthRtop1}
%
%\displayonelarge{Post-Selection/Truth-Level Correlation}{
%    At first glance, truth performance appears only loosely correlated to post-selection performance.
%    \vspace{4mm}
%
%    But, performance can be made more similar by capping the statistics of the truth samples
%}{Nweight_integral_VS_Nweight_uncapped_truth_integral}
%
%
%\displayfour{Post-Selection Performance (Top) VS ``Capped" Truth-Level (Bottom)}
%{negative_weights_toprank0}
%{negative_weights_toprank1}
%{negative_weightstruthRtop0}
%{negative_weightstruthRtop1}
%
%
%\displayonelarge{Post-Selection/Capped-Truth-Level Correlation}{
%    Artificially limiting truth samples to 10\% of their events provides a much stronger correlation between truth and post-selection performance
%}{Nweight_integral_VS_Nweight_truth_integral}
%

    \announcesection{Effective Statistics}
\displayone{Effective Statistics}{
    Effective Statistics also acts as a similar measure of ``closeness''.
    \vspace{3mm}

    \textit{Effective Stats},
    $\epsilon \equiv \frac{ (\sum\limits_{i=1}^6 c_i\sigma_i)^2 }{ \sum\limits_{i=1}^6 (c_i\sigma_i)^2 } $
    \vspace{3mm}

    Unlike solidarity, effective stats \textit{heavily}
    penalizes negative coefficients and large cancellations between samples.

}
{effstat_performance_kv1}

\displaytwocaption{Effective Statistics VS Solidarity}{
    Effective Stats favors the upper-right region over all else.
    However, truth simulations of samples in this region produce mediocre performance gains,
        in line with the predictions of solidarity.
}
{solidarity_performance_kv1} {Solidarity}
{effstat_performance_kv1} {Effective Statistics}

\displaythree{Performance Predictions with Different $\kv$}{
    Produces similar, though not identical, results as solidarity
}
{effstat_performance_kv0.5}
{effstat_performance_kv1}
{effstat_performance_kv1.5}


\end{document}
