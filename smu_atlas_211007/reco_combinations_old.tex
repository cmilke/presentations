\section{Applying Linear Combination}

\frame{
    \frametitle{Linear Combination Using Post Reconstruction/Selection Samples}
    \begin{columns}
        \begin{column}{0.4\textwidth}
            \begin{center} 
            {\tiny Signal Sample Basis Set}

            \resizebox{0.2\textheight}{!}{ \begin{tabular}{ |l|l|l| }
                \hline
                \textbf {$\kappa_{2V}$} & \textbf {$\kappa_\lambda$} & \textbf {$\kappa_V$} \\
                \hline
                    1.  &   1. & 1.  \\
                    2.  &   1. & 1.  \\
                    1.5 &   1. & 1.  \\
                    0.  &   1. & 0.5 \\
                    1.  &   0. & 1.  \\
                    1.  &  10. & 1.  \\
                \hline
            \end{tabular}}
            \end{center}

        \end{column}
        \begin{column}{0.6\textwidth}
            \resizebox{0.8\textwidth}{!}{ \begin{minipage}{1.0\textwidth}
            Linear Combination Equation

            \vspace{10mm}

            {\tiny \input{final_amplitude_current_3D_reco.tex}}
            \end{minipage}}
        \end{column}
    \end{columns}
}

\displaythree{Validity of Sample Combinations}
{ \small 
    Combining the six basis samples allows for modelling of distributions for any coupling values.
}
{reco_mHH_cvv1p0cl2p0cv1p0_ancient}
{reco_mHH_cvv0p0cl1p0cv1p0_ancient}
{reco_mHH_cvv4p0cl1p0cv1p0_ancient}
