%Imports and customization
\usepackage{tikz}
\usepackage{graphicx}
\graphicspath{ 
    {./images/}
}

\beamertemplatenavigationsymbolsempty
\setbeamertemplate{sidebar right}{}
\setbeamertemplate{footline}{
    \hfill\usebeamertemplate***{navigation symbols}
    \hspace{1cm}\insertframenumber{}/\inserttotalframenumber
}
\setbeamertemplate{caption}{\raggedright\insertcaption\par}
\setbeamersize{text margin left=5mm,text margin right=5mm} 

\setbeamerfont{itemize/enumerate body}{size=\scriptsize}
\setbeamerfont{itemize/enumerate subbody}{size=\scriptsize}
\setbeamerfont{itemize/enumerate subsubbody}{size=\scriptsize}


%Custom Macros
\newcommand{\statwarn}{
    \tiny \color{red} Absolute numbers here mean NOTHING. Plots are based on small (100k events) samples, and are highly biased. All that matters is relative position!
}

\newcommand{\inputvardisplaytwo}[2]{
    \frame{
        \foreach \inputvar in {#1, #2} {
            \begin{columns}
                \foreach \jettype in {0, 1, 2} {
                    \begin{column}{0.33\textwidth}
                        \begin{figure}
                            \includegraphics[width=\linewidth,height=\textheight,keepaspectratio]{training_input_study/plot_\inputvar_\jettype}
                        \end{figure}
                    \end{column}
                }
            \end{columns}
        }
    }
}

\newcommand{\inputvardisplay}[3]{
    \frame{
        \foreach \inputvar in {#1, #2, #3} {
            \begin{columns}
                \foreach \jettype in {0, 1, 2} {
                    \begin{column}{0.33\textwidth}
                        \begin{figure}
                            \includegraphics[width=\linewidth,height=\textheight,keepaspectratio]{training_input_study/image_\inputvar_\jettype}
                        \end{figure}
                    \end{column}
                }
            \end{columns}
        }
    }
}


\newcommand{\fullscreenimage}[2]{
    \frame{
        \frametitle{#1} 
        \begin{figure}
        \includegraphics[width=\linewidth,height=\textheight,keepaspectratio]{#2}
        \end{figure}
    }
}


\newcommand{\displayone}[3]{
    \frame{
        \frametitle{#1} 
        \begin{columns}
            \begin{column}{0.5\textwidth}
                #2
            \end{column}
            \begin{column}{0.5\textwidth}
                \begin{figure}
                    \includegraphics[width=\linewidth,height=\textheight,keepaspectratio]{#3}
                \end{figure}
            \end{column}
        \end{columns}
    }
}


\newcommand{\displaytwo}[4]{
    \frame{
        \frametitle{#1} 
        #2
        \begin{columns}
            \begin{column}{0.5\textwidth}
                \begin{figure}
                    \includegraphics[width=\linewidth,height=\textheight,keepaspectratio]{#3}
                \end{figure}
            \end{column}
            \begin{column}{0.5\textwidth}
                \begin{figure}
                    \includegraphics[width=\linewidth,height=\textheight,keepaspectratio]{#4}
                \end{figure}
            \end{column}
        \end{columns}
    }
}


\newcommand{\displaythree}[5]{
    \frame{
        \begin{columns}[T]
            \begin{column}{0.5\textwidth}
                \insertframetitle{#1}\\
                #2
            \end{column}
            \begin{column}{0.5\textwidth}
                \begin{figure}
                    \includegraphics[width=\linewidth,height=\textheight,keepaspectratio]{#3}
                \end{figure}
            \end{column}
        \end{columns}
        \begin{columns}[T]
            \begin{column}{0.5\textwidth}
                \begin{figure}
                    \includegraphics[width=\linewidth,height=\textheight,keepaspectratio]{#4}
                \end{figure}
            \end{column}
            \begin{column}{0.5\textwidth}
                \begin{figure}
                    \includegraphics[width=\linewidth,height=\textheight,keepaspectratio]{#5}
                \end{figure}
            \end{column}
        \end{columns}
    }
}


\newcommand{\vardisplay}[2]{ \displaythree{#2}{}{#1_0}{#1_1}{#1_2} }


\newcommand{\announcesection}[1]{
    \section{#1}
    \frame{
        \begin{center}
            {\huge #1} 
        \end{center}
    }
}
